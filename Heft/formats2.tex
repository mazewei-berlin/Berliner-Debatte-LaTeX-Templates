%
% Barrierefreie PDF
%
%Metadaten aus bdi.mpxdata Datei in Dokument einfügen
\usepackage[a-3u]{pdfx}

%\usepackage[tagged]{accessibility}
%Struktur im PDF speichern
\usepackage[highstructure]{accessibility}

%Einbetten der Schriftarten 
\usepackage{luacode}
\begin{luacode}
    local function embedfull(tfmdata)
    tfmdata.embedding = "full"
    end
    
    luatexbase.add_to_callback("luaotfload.patch_font", embedfull, "embedfull")
\end{luacode}

%
% unterer randausgleich
%
\flushbottom

\usepackage[T1]{fontenc}
\usepackage{lmodern}
\usepackage[ngerman]{babel}
\usepackage{microtype}
%\usepackage[showframe]{geometry}
%\usepackage{layouts}
\usepackage[german=quotes]{csquotes}

%
%text durchstreichen
%
\usepackage{soulutf8}

%
% Formatierung von URLs und interner Verlinkung der PDF
%
%\usepackage[urlcolor=black]{hyperref}
\usepackage{hyperref}
\PassOptionsToPackage{urlcolor={black}}{hyperref}

%\usepackage{showframe}% zum Anzeigen des Seitenlayouts
% optischer Randausgleich/microtypografie
\usepackage{%
    microtype,%
    ellipsis,%
    mparhack,
    parskip
}

%
% TOC
%
\KOMAoption{toc}{ 
    chapterentrydotfill,
    listof,
    indexnumbered }
\setkomafont{chapterentry}{\normalfont}
\setcounter{tocdepth}{0}

\deftocheading{toc}{}%

% Chapter zahlen entfernen
\renewcommand*\chapterformat{}
\renewcommand*\sectionformat{}
\RedeclareSectionCommand[
%toclinefill=\TOCLineLeaderFill,
%tocnumwidth=0pt,
afterindent=false, 
%beforeskip=0pt,
afterskip=1\baselineskip,
beforeskip=.5\baselineskip,
tocentrynumberformat=\gobbleentrynumber
]{chapter}
\newcommand*\gobbleentrynumber[1]{}

% Zweispaltigkeit
\usepackage{multicol}
\setlength{\columnsep}{5mm}

\usepackage{caption}
%\DeclareCaptionLabelFormat{bf-parens}{(\textbf{#2})}
\captionsetup[table]{
    labelformat=empty,
}

%
% SCHRIFT
%
\usepackage{fontspec}
\setmainfont{WarnockPro-Regular.otf}
[
BoldFont = WarnockPro-Bold.otf,
ItalicFont = WarnockPro-It.otf,
]
\newfontfamily{\chapterfont}{WarnockPro-Bold.otf}
\newfontfamily{\sectionfont}{WarnockPro-Bold.otf}

% Heft Titel über Inhaltsverzeichnis
\newcommand{\bdititle}[1]{\huge #1}
\newcommand{\bdisubtitle}[1]{\LARGE #1}

%
% KONFIGURATION DER Schriftarten 
%
% Konfigurieren der Schriftgrößen im Allgemeinen
\makeatletter
\renewcommand\normalsize{\@setfontsize\normalsize{10}{11}}
\renewcommand\large{\@setfontsize\large{10}{11}}
\renewcommand\LARGE{\@setfontsize\LARGE{18}{21.6}}
\renewcommand\huge{\@setfontsize\huge{20}{20}}
\makeatother
%Schriftgröße Kapiteltitel und Zwischenüberschrift
\addtokomafont{chapter}{\chapterfont\huge}
\addtokomafont{section}{\chapterfont\large}

%Platz um Absätze
%Einrückung erste Zeile eines Absatze
\setlength{\parindent}{1em}

%Abstand nach Absatz%
%\RedeclareSectionCommands[
%beforeskip=4em,
%runin=true,
%afterskip=40em
%]{paragraph,subparagraph}

% Konfigurieren des Abstandes nach der Artikel zum eigentlichen Text
\newcommand{\titlespaceAfter}{\parskip}

% Kommando für Editorialüberschrift
% Argument: Autor(en)
\newcommand{\bdieditorial}[1]{
    \fancyhead[RE]{\bdiReleaseTitle} % TODO(Aurel): Into command
    \fancyhead[LO]{Editorial}
    
    \chapter[
    head={
        Editorial
    },
    tocentry={
        \textit{#1}\\
        Editorial
    }
    ] {\bigskip{\fontsize{14}{0}\selectfont {#1}}\\\bigskip Editorial}%
    \vspace{\titlespaceAfter}
}

% Kommando für Chapters: Setzt Author, Titel und Untertitel und den header automatisch.
% erstes Argument: Autor(en)
% zweites Argument: Titel
% optionales drittes Argument: formatierter Titel: falls extra Formatierung nötig ist
% optionales viertes Argument: Untertitel
% NOTE: nicht genutzte optionale Argumente einfach leer lassen!
\newcommand{\bdichapter}[4]{
    \fancyhead[RE]{#1}
    \fancyhead[LO]{\leftmark}
    
    \chapter[
    head={
        {#2}
    },
    tocentry={
        \textit{#1}\\
        {#2}
    }
    ] {\bigskip{\fontsize{14}{0}\selectfont {#1}}\\ {%
            % Falls kein extra formatierter Titel gegeben wird, nutze den
            % einfachen Titel (Argument 2)
            \ifx \relax#3\relax%
            #2%
            \else%
            #3%
            \fi%
    }}%
    % Untertitel, falls gegeben
    \ifx \relax#4\relax%
    \else%
    {\fontsize{14}{0}\selectfont #4}\bigskip
    \fi%
    %\vspace{\titlespaceAfter}
}

% Vor dem ersten Aufruf ist \newpage nötig
% erstes Argument: Autor(en)
% zweites Argument: Titel
% drittes Argument: Rezensiert von
% Beispiel:
%\bdirezension{Anne Hartmann,\\Reinhard Müller (Hg.)}{Tribunale als Trauma}{Wladislaw Hedeler}
%\usepackage{etoolbox}
\newcommand{\bdirezension}[3]{
    \fancyhead[RE]{Besprechungen und Rezensionen}%
    \fancyhead[LO]{\bdiReleaseTitle}%
    {
        % NOTE: auskommentieren um es auf der nächsten Halbseite zu starten
        %\vfill\null\columnbreak
        \vspace{\titlespaceAfter}
        \chapter[
        head={
            {#2}
        },
        tocentry={
            {#1}:\\
            {#2}\\
            Rezensiert von \textit{#3}
        }
        ] {\bigskip{\fontsize{14}{0}\selectfont {#1}:}\\\bigskip {%
                {#2}\\%
                {\fontsize{12}{0}\selectfont Rezensiert von {#3}}
        }}%
    }%
    \vspace{.5\titlespaceAfter}
}

% Zeilenabstand Blocksatz
\linespread{1.1}
\setlength{\baselineskip}{1.1pt}

%
% SEITENHEADER
%

\usepackage[headings]{fancyhdr}
\pagestyle{fancy}
\renewcommand{\headrulewidth}{1pt}
\renewcommand{\chaptermark}[1]{%
    \markboth{#1}{}}
\fancyhead{} % clear all header fields
\fancyhead[LE]{\thepage}
\fancyhead[RO]{\thepage}

\fancyheadoffset[RE,LO]{0\textwidth}

% head über einem neuen Kapitel
\fancypagestyle{plain}{%
    \fancyhf{}%
    \fancyhead[LO]{\bdiReleaseTitle}
    \fancyhead[RE]{\bdiReleaseTitle}
    
    \renewcommand{\headrulewidth}{1pt}% Line at the header invisible
}

%
% FUSSNOTEN am Ende
%
\usepackage{enotez}
\setenotez{reset=true}

\newlength{\normalparindent}
\AtBeginDocument{\setlength{\normalparindent}{\parindent}}

\newcommand{\myendntefnt}{\fontsize{8}{0}}
\DeclareInstance{enotez-list}{custom}{paragraph}
{
    heading = \section*{#1} ,
    notes-sep = \parskip,
    format = \footnotesize\leftskip\parindent,
    number = \makebox[10pt][r]{\myendntefnt{#1}}
}

%
% LISTEN und BIBLIOGRAFIE
%
\makeatletter
\newenvironment{bibdescription}{%
    \@nameuse{fontitem\romannumeral\the\@itemdepth}
    
    \begin{description}[leftmargin=0pt, itemindent=.4cm, nosep]
    }{%
    \end{description}
}
\makeatother
\newcommand{\fontitem}{\footnotesize}
\usepackage{enumitem}

%
% BILDER
%
\usepackage{graphicx}
\graphicspath{../Bilder/}

\usepackage[
labelformat=empty,
format=plain,
labelformat=simple,
font=it,
]{caption}
\counterwithout{figure}{chapter}
\usepackage{wrapfig}

%
% tables
%
\usepackage{longtable}
\usepackage{array}
\usepackage{vcell}

%
% GEOMETRIE der Seiten
%
\usepackage{geometry}
\geometry{
    twoside,
    paperwidth=162mm,
    paperheight=227mm,
    width=137mm,
    top=20mm,
    bottom=12mm,
    inner=13mm,
    outer=12mm,
    %height=195mm,
    %showframe=true,
}