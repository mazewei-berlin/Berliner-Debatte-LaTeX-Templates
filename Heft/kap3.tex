\bdichapter{Dieter Segert}{Die Krise der Linken in Osteuropa und ihre globalen Wurzeln}{}{}

\begin{multicols*}{2}
    \noindent 2018 fand in dem renommierten Belgrader Institut für Sozialwissenschaften eine Konferenz statt, die sich mit der Krise der Linken in der Region wie auch europaweit beschäftigte.\endnote{Die Veröffentlichung der Konferenzbeiträge erfolgte erst später: siehe Ristic 2021.}  Sie trug den schönen Titel: „Die Linke ist tot, es lebe die Linke!“ Der analoge Ruf – bezogen auf den gestorbenen alten und den ihm nachfolgenden neuen König – ist aus der Geschichte Europas bekannt. Aber in Bezug auf die Linke entstehen bei seiner Verwendung Fragen wie die folgenden: Was ist die alte Linke? Worin bestand ihre Krise? Wer folgt ihr nach? 

    \section{Osteuropa als Menetekel der Linken}

\noindent In Osteuropa begann und endete das „kurze 20. Jahrhundert“. Hier – in Sarajevo – fand die Initialzündung zur Katastrophe des 20. Jahrhunderts statt, zum Ersten Weltkrieg, hier (in Moskau) wurde eine neue Abteilung der radikalen Linken geboren, die kommunistische Weltbewegung, gut 70 Jahre später, 1989–1991, stürzte ihr wichtigstes Resultat, die Sowjetunion und der sowjetische Staatssozialismus, krachend ein. Dies hatte Auswirkungen weltweit, nicht nur auf die Erben der Komintern, sondern auch die der Zweiten Internationale.\par

Warum war das so? Seit Jahrzehnten, seit 1945, hatte es einen Wettbewerb zwischen dem sozial gebändigten Kapitalismus und dem 1917 geborenen Staatssozialismus gegeben. Die Zeitenwende 1989, die in den Zerfall der Sowjetunion mündete, hatte so ausgesehen, als ob damit jegliche systemische Alternative zum Kapitalismus gescheitert wäre. Dem „realen Sozialismus"\endnote{Der „reale Sozialismus“ war ein Kampfbegriff der herrschenden Ideologie im Osten vor 1989 gegen alternative Sozialismuskonzepte anderer Linker. Er sollte ausdrücken, dass man sich zwar alles mögliche Schöne ausdenken könne, aber dass nur im sowjetischen Herrschaftsbereich der Kapitalismus wirklich überwunden worden sei.}  war die Zukunft abhandengekommen. Der Spruch Fukuyamas vom „Ende der Geschichte“ (1992) hatte Konjunktur. \par

Es blieb nicht so. Am Ende des ersten Jahrzehnts des neuen Jahrtausends schien sich die Geschichte erneut zu drehen. Die weltweite Finanzkrise der Jahre ab 2007/08 erweckte die Kritik am vorherrschenden Wirtschaftssystem erneut. In der wissenschaftlichen und journalistischen Debatte wurde Fukuyamas These erneut diskutiert und seziert. Ther sprach vom „anderen Ende der Geschichte“ (Ther 2019). Aber es gab weitere Argumente gegen den endgültigen Sieg der westlichen Ordnung und ihrer Werte, etwa den wirtschaftlichen Aufstieg Chinas bei Beibehaltung seiner politischen Ordnung oder den politischen Islam. Und im Innern der westlichen Welt breitete sich eine antiwestliche politische Strömung aus, sie wurde als „rechter Populismus“ gekennzeichnet. \par

Osteuropa spielte in dieser Gegenbewegung wieder eine besondere Rolle. Hier war der Sieg des kapitalistischen Westens über seine alternative Ordnung besonders überzeugend gewesen, gerade deshalb wäre anzunehmen, dass seine Krise von 2008 eine Renaissance linker Politik bewirken würde. Es kam anders. Der Nachhall der schmerzhaften wirtschaftlichen Umbaumaßnahmen im Verlaufe der 1990er Jahre führte nicht zum Wiederaufstieg der politischen Linken, sondern zu dem des rechten Populismus, für den besonders Orbáns Fidesz und Kaczynskis PiS stehen.\par

Warum das so war und ob das so bleiben muss, soll nachfolgend diskutiert werden. 

\section{Linke politische Parteien in Osteuropa: ein kurzer historischer Abriss}

\noindent Ausgangspunkt ist die Zeit nach 1945. Die politische Linke bestand ursprünglich in vielen Ländern aus Sozialdemokraten und Kommunisten. In einigen Ländern existierten auch Bauernparteien mit einem linken Potential, so in Bulgarien und Ungarn. Nach Beginn des Kalten Krieges ab 1947 kam es zu einer Einebnung aller nichtkommunistischen linken Parteien. Die Sozialdemokraten fusionierten unter politischem Druck mit den kommunistischen Parteien, wobei kleine Gruppen in den Westen gingen, um dort Exilparteien zu bilden. Die linken Bauernparteien lösten sich ebenfalls auf oder wurden zu willigen Vollstreckern der kommunistischen Politik. In einigen Krisen des Staatssozialismus gründeten sich die sozialdemokratischen Parteien auch in jenen Ländern neu, so etwa in der Tschechoslowakei im Jahr 1968. Diese Gründungen überlebten aber das Ende der jeweiligen staatssozialistischen Krisen nicht. 

In der finalen Krise des sowjetischen Staatssozialismus, ab Ende der 1980er Jahre, kam es zur Wiedergeburt des Parteienpluralismus; in einigen Fällen wurden auch historische linke, christdemokratische und Bauernparteien wiedergegründet. Dabei spielten teilweise Personen, die aus dem Exil zurückkehrten, eine Rolle. Die sozialdemokratischen Parteien, die sich um die kleinen Gruppen von Exilpolitikern bildeten, hatten allerdings meist keinen politischen Erfolg. In mehreren Ländern, zuerst in Ungarn und Polen, transformierten sich jedoch die kommunistischen Staatsparteien erfolgreich zu Mitgliedsparteien der europäischen Sozialdemokratie. Ab Mitte der 1990er Jahre waren sie dann auch im demokratischen Parteienwettbewerb erfolgreich. Die parallel dazu neuentstandenen sozialdemokratischen Traditionsparteien hingegen hatten kaum in einem Land nennenswerten politischen Einfluss. Die tschechoslowakische Sozialdemokratie (später die tschechische) war eine Ausnahme. In ihr fusionierten Exilsozialdemokraten mit Reformkommunisten (die ihre Tradition in den Reformen des Prager Frühlings hatten). Miloš Zeman war einer jener Postkommunisten von 1968, die der sozialdemokratischen Partei, der ČSSD, 1998 erstmals zum Regierungserfolg verhalf. 

Generell gilt: Nach einer Phase der Delegitimierung der Nachfolgeparteien der Staatsparteien aus der Zeit des sowjetischen Staatssozialismus machten die meisten von ihnen nach ihrer Wandlung hin zu einem sozialdemokratischen Programm ab Mitte der 1990er Jahre Regierungserfahrungen. Diese Phasen einer Führung von Regierungen bzw. als Juniorpartei von Koalitionsregierungen lagen in Ostmitteleuropa zwischen 1994 und 2010. Danach kam es zu deutlichen Einflussverlusten dieser Parteien und im Extrem sogar zu ihrem Ausscheiden aus dem Parlament. In Belarus, der Ukraine und Russland nahm die Entwicklung einen anderen Verlauf: Die kommunistischen Staatsparteien wurden zwar in einer bestimmten Phase auch „sozialdemokratisiert“, aber sie spielten keine zentrale Rolle für die politische Herrschaft. Es bildeten sich Patronageregime heraus.\endnote{Im Folgenden wird deshalb unter „Osteuropa“ der Raum der neuen EU-Mitgliedstaaten und der Kandidatenstaaten in Südosteuropa verstanden. }
\end{multicols*}

\begin{longtblr}[
    caption = {Erfolge und Niederlagen Sozialdemokratischer Parlamentsparteien in Osteuropa},
  ]{
    width = \linewidth,
    colspec = {Q[200]|Q[183]|Q[206]|Q[181]|Q[158]},
    cells = {t},
    rowhead = 1
  }
  {
    Land/
    Partei \\als Mitglied der \\SPE* (bzw. \\\textbf{Assoziiert/}\\\textbf{Beobachter-}\\status)\endnote{Quelle: http://www.parties-and-elections.eu/, eigene Recherchen.\\
    **Abkürzungen: BSP: Bulgarische Sozialistische Partei; ČSSD: Tschechische Sozialdemokratische Partei; DK (ab 2011): Demokratische Koalition (Ungarn); DP: Demokratische Partei (Serbien); DPS: Demokratische Partei der Sozialisten (Montenegro); LDDP: Demokratische Arbeiterpartei Litauens; LSDP (ab 2001): Sozialdemokratische Partei Litauens; LSSP: Lettische Sozialdemokratische Arbeiterpartei; LVV: Bewegung für Selbstbestimmung (Kosovo); NISMA: Sozialdemokratische Initiative (Kosovo); MSZP: Ungarische Sozialistische Partei; PBS: Partei der Bulgarischen Sozialdemokraten; PDSR: Partei der Sozialen Demokratie Rumäniens; PSD (ab 2001): Sozialdemokratische Partei (Rumänien); PSSh: Sozialistische Partei Albaniens; SD (ab 2005): Sozialdemokraten (Slowenien); SDL‘: Partei der Demokratischen Linken (Slowakei); SDE: Sozialdemokratische Partei (Estland); SDP: Sozialdemokratische Partei Kroatiens; SDPS: Sozialdemokratische Partei „Harmonie“ (Lettland); SDSM: Sozialdemokratische Union Mazedoniens; SLD: Vereinigung der Demokratischen Linken (Polen); Smer-sd (ab 1999): Richtung-Sozialdemokratie (Slowakei); SPE: Sozialdemokratische Partei Europas; UP: Arbeitsunion (Polen); ZLSD: Vereinigte Liste der Sozialdemokraten (Slowenien).\\
    *** Stand Ende Juli 2022.\\
    **** Im Kosovo gibt es keine Partei, die aktuell mit der SPE in irgendeiner formalisierten Beziehung steht, allerdings definieren sich zumindest drei Parteien (zwei davon sind bei Parlamentswahlen relevant) innenpolitisch als sozialdemokratisch, die Lȅvizja Vetȅvendosje und die NISMA (Sozialdemokratische Initiative). 
    }
    } & {
    1.
    Regierungs-\\teilnahme nach \\1989 (führend=PM \\oder in Koalition= K)
    } & {
    Folgende \\Regierungs-\\teilnahme 
    \\(Jahre)
    } & {
    Absturz
    / großer Verlust, Jahr: \\alte - folgende\\~Stimmenzahl \\(Prozent)
    } & {
    Gegenwärtige \\Stellung zur \\Regierung**
    \\~
    }\\
    \hline
  Polen/SLD (+UP) & 1995–1997 (PM) & 2001–2005 (PM) & {
    Großer Verlust
    \\2005: 41-11
    \\2015: keine \\Mandate
    \\2019: - 12,6
    } & Opposition\\
    \hline
  {
    Slowakei/\\SDĽ-smer-sd
    } & 1998–2002
    K (SDĽ) & {
    2000–2010;
  \\~2012–2020: \\PM (smer-sd)
    } & {
    Absturz SDĽ
    \\2002: 15-1
    \\GroßerVerlust \\(smer-sd)
    \\2016: 44-28
    \\2020: -18
    \\Spaltung 09\_2020 \\(HLAS sd)
    } & {
    Gespalten
    und in\\ Opposition
    }\\
    \hline
  {
    Tschechien/\\ČSSD
    } & {
    1998–2006 PM \\(1998–2002: \\Minderheits-r\\egierung)
    } & {
    2014–2017:
    PM
    \\2017
    bis 2021: K
    } & {
    Großer Verlust
    \\2010: 32-22
    \\2017: 21 - 7
    \\2021: 7 - 5
    } & {
    Außer-\\parlamen-\\tarische \\Opposition 
    }\\
    \hline
  {
    Ungarn/\\MSZP (+DK)
    } & 1994–1998 PM & 2002–2010 PM & {
    Großer Verlust
    \\2010: 41-19
    \\2018: 26- 12 \\(MSZP)
    \\- 5 (DK)
    } & {
    Gespalten
    und in\\ Opposition
    }\\
    \hline
  {
    Rumänien/\\PDSR…PSD
    } & 1992–1996 PM & {
    2000–2004 PM
    \\2012–2015 PM
    \\2016–2019 PM
    } & {
    Großer Verlust
    \\2020: 46 - 29
    } & Koalitionspartner\\
  {
    Bulgarien/BSP \\(PBS: A)
    } & {
    1990–1991 PM
    \\(BSP)
    } & {
    1995–1997 PM
    \\2005–2009 PM
    \\2013–2014 K
    \\2021/11–2022/6\\K
    } & {
    Oszillieren
    \\1997:
    44-22
    \\2009:
    31-18
    \\2014:
    27-15
    \\2021
    1/2: 27\\-15/13
    \\2021
    3: 10
    } & {
    Nach
    2022/6 \\Regierungs-\\neubildung 
    \\(Machtteil-\\nahme seit\\ 2017 durch \\Präsidenten, \\auf Liste BSP \\gewählt)
    }\\
    \hline
  {
    Slowenien/\\ZLSD…SD
    } & {1992–2004 \\(in Koalition)} & {
    2008–2012 (PM)
    \\~
    } & {
    Verluste
    \\2011: 31-11
    \\2014: 11-6
    \\2018: 6-10
    \\2022: 7 
    } & Koalitionspartner\\
    \hline
  Kroatien/SDP & 2000–2003 PM & 2011–2016 (PM) & {
    Verluste: 
    \\2003: 41-23
    \\2020: 34 - 25 \\(im Bündnis)
    } & Opposition\\
    \hline
  Serbien/DP & {
    2000–2003 PM
    \\
    } & {
    2007­–2008
    \\2003–2012: Präsident
    } & {
    2008: 23-38
    \\Verlust
    \\2014: 22-6
    \\2020: 0
    } & {
    Außer-\\parlament-\\arische \\Opposition
    }\\
    \hline
  {
    Montenegro/\\DPS (A)
    } & 1990–2020 PM & ~ & {
    Stets 40-51 \%
    \\2020: 41 - 35
    } & Präsident\\
    \hline
  {
    Kosovo***/LVV, \\NISMA
    } & 2019 PM & Ab 2021 PM & {
    Starker Anstieg
    \\2021: 26 - 50
    } & {
    Stärkste\\Regierungs-\\partei
    }\\
    \hline
  Albanien/PSSh (A) & 1991 PM & {
    1991,
    1997–2005, 
    \\seit
    2013 PM
    } & {
    Absturz
    \\2005: 42 - 9
    \\Anstieg
    \\2009: 9 - 40
    \\2021: 48-49
    } & {Allein-\\regierung}\\
    \hline
  {
    Nordmazedonien/\\SDSM (A)
    } & {
    1992-1998 PM
    \\
    } & {
    2004­2–008 PM
    \\seit 2017 PM
    } & {
    Zw. 24 und 49 \\Prozent,
    \\2020: 36
    } & {
    PM in \\Koalitions-\\regierung
    }\\
    \hline
  Litauen/LDDP-LSDP & 1993-1996 PM & {
    2001–2006 (PM)
    \\2012-2016 (PM)
    \\2016–2020 - K
    } & {
    Verluste: 
    \\2000: 
    \\31- 20; 2008: \\21-12
    \\2020: 14 - 9
    } & Opposition\\
    \hline
  {
    Lettland/
    SDPS + 
    \\LSSP
    (B)
    } & ~ & ~ & {
    SDPD:
    bis 2006 \\unter 10 seitdem \\20 und mehr \\Prozent
    } & {
    (Stärkste \\Partei, aber in) 
    \\Opposition
    }\\
    \hline
  Estland/ SDE & {
    2007–2011 K
    \\~
    } & {
    2014–2015 K
    \\2016–2019 K
    } & {
    Verlust: 2003: 15-7
    \\2019: 
    \\15 - 10
    } & Opposition
    \hline
  \end{longtblr}

\begin{multicols*}{2}

    \noindent Diese Entwicklung erscheint auf den ersten Blick geradezu paradox, da ja in der globalen Finanzkrise ab 2008 die Krisenhaftigkeit des neoliberalen Finanzkapitalismus deutlich zu Tage trat und eine sozialstaatlich gemilderte Variante des Kapitalismus – für die bekanntlich die Sozialdemokratie stand – auf die Tagesordnung zu treten schien. So war es aber nicht, auch nicht in Westeuropa. In Osteuropa jedoch wurde dieses Paradoxon besonders deutlich. Im Sommer 2022 regierten die Sozialdemokraten in keinem Land Ostmitteleuropas, außer in Slowenien. In Südosteuropa war die Bilanz etwas positiver. Hier verfolgen die „Sozialdemokraten“ allerdings eine Mischung aus sozialdemokratischer und nationalistisch-populistischer Programmatik.

    \section{Der Modus der post-sozialistischen Transformation}

    \noindent Die Paradoxie löst sich auf, wenn wir in die jüngste Geschichte zurückblicken. Die wichtigste Ursache für den gegenwärtigen Tiefstand des linken Einflusses in Osteuropa liegt in der Art des Systemwechsels von 1989 begründet. Die existierende staatssozialistische Ordnung war in eine tiefe Krise geraten und brach zusammen. An die Stelle der Diktaturen traten demokratische Ordnungen. Die vorherigen kommunistischen Staatsparteien reformierten sich überwiegend hin zu sozialdemokratischen, eine kleine Zahl, vor allem in Südosteuropa, wandelte sich zu nationalistischen Parteien (Bozóki, Ishiyama 2002). Der gleichzeitig vollzogene Wechsel des Wirtschaftsmodells und der sozialen Ordnung stellte jedoch eine sozialdemokratische Politik des Schutzes der Beschäftigten und der Unterstützung des sozialen Ausgleichs vor schwierige Aufgaben. Mehr noch, die reformierten Postkommunisten wurden als Sozialdemokraten zu einem der hauptsächlichen Gestalter einer Transformation, in der eine Bevölkerung, die 1989 eine Verbesserung ihrer Lebenssituation erwartet hatte, stattdessen den Härten eines nahezu unregulierten Kapitalismus\endnote{Ristic (2021: 15) spricht vom „Modell eines deregulierten Kapitalismus“.}  ausgesetzt wurde. Die Sozialstaatlichkeit des Ancien Regime wurde abgebaut, die neuen sozialstaatlichen Institutionen waren schwach. Dazu trug auch die Schwäche der Gewerkschaften bei, die durch ihre staatsnahe Stellung in der vorangegangenen Ordnung an Ansehen verloren hatten.

    Zwar hätte der Zusammenbruch des Staatssozialismus als eine Bestätigung des sozialdemokratischen Gegenmodells zum kommunistischen Weg verstanden werden können, real aber führte die Delegitimierung des staatssozialistischen Modells in eine Krise der politischen Linken insgesamt, in all ihren Schattierungen. Mit dem sowjetischen Sozialismus erodierte überhaupt die Idee, dass es eine grundsätzliche Alternative zum bestehenden kapitalistischen Wirtschafts- und Gesellschaftsmodell geben könnte. Nur noch innerhalb der gegebenen Ordnung schienen Verbesserungen möglich, und auch die erforderten einen breiten Konsens. Hinzu kam die spezifisch sozialdemokratische Variante des neoliberalen Gesellschaftsmodells, der sogenannte „Dritte Weg“, von der Politik des US-Präsidenten Clinton angeregt und durch Blair (Labour) und Schröder (SPD) ausformuliert.\endnote{  Das Ziel jener Ideologie war ein schlanker Staat mit reduzierten Staatsausgaben, was zu Lasten der bestehenden Sozialsysteme gehen musste. Der Wirtschaftsaufschwung setzte zudem auf niedrige Unternehmenssteuern. Damit war die Vorstellung verbunden, dass der im Gefolge einer boomenden Unternehmensentwicklung entstehende Reichtum stetig nach unten durchtröpfeln würde, wodurch eine Anhebung des allgemeinen Wohlstands erwartet wurde. Das neoliberale Programm scheiterte in Osteuropa jedoch deutlicher und schneller als im Westen des Kontinents.}

    Die Erosion der Systemalternative zum Kapitalismus hatte bereits vor 1989, in den 1970er Jahren begonnen, als die kommunistischen Parteien im sowjetischen Herrschaftsbereich den Weg des Wettbewerbs um die bessere Lösung der globalen Lebensfragen mit dem Westen einschlugen. Das zeigte sich in der Konkurrenz zweier Modelle des Sozialstaates in West- und Osteuropa (Boyer 2008). Für das östliche Modell wurde der Begriff „Konsumsozialismus“ (Staritz 1996; Hertle 2006) geprägt. Er stellt den Versuch einiger Staaten des sowjetischen Sozialismus dar, bestimmte Momente der Anziehungskraft des sozial regulierten Kapitalismus auf die Lohnarbeiterschaft zu kopieren. Und in diesem Sinne war es auch ein stilles Eingeständnis dessen, dass der Versuch, eine umfassende Alternative zum Kapitalismus zu schaffen, gescheitert war. Unter Chruschtschow gab es etwa noch das Ziel, den privaten PKW-Verkehr durch ein anderes, öffentliches Verkehrssystem zu ersetzen. Unter Breshnew, Honecker, Ceaușescu und Husák hingegen wurde eine eigene PKW-Produktion aufgebaut. Dabei häufig auf Grundlage von Kooperationen mit westlichen Automobilkonzernen wie Fiat, Volkswagen oder Renault. 
    Der Staatssozialismus verlor den Systemwettbewerb nicht nur wegen seiner geringeren Wirtschaftskraft, sondern auch deshalb, weil er im Alltag kein überzeugendes Gegenmodell zur westlichen Konsumgesellschaft schaffen konnte. In der DDR zeigte sich diese mangelnde Attraktivität auch darin, dass die westdeutsche Mark zur Traumwährung der ostdeutschen Bevölkerung avancieren konnte. Wichtiger noch waren aber Wandlungen der Zielvorstellungen der staatssozialistischen Eliten in einigen der Staaten im Verlaufe der 1980er Jahre, vor allem in Ungarn, Polen und der Sowjetunion, die sich damit den jugoslawischen Reformkonzepten annäherten: Sie strebten nach einer Verbindung von Marktsozialismus und Demokratie, die sich allerdings schnell als nicht machbar erwies.

Daneben spielte es eine Rolle, dass der westliche Sozialstaat bis zum Aufstieg des neoliberalen Wirtschaftsmodells zwar keine umfassende soziale Gleichheit schaffen konnte und das auch nicht anstrebte, aber zumindest eine gewisse Art von sozialer Sicherheit für die meisten Lohnabhängigen herstellte. Durch Arbeitslosenversicherung, Altersrenten und ein ausgebautes Gesundheitssystem war in den kontinentaleuropäischen Ländern ein System entstanden, das sich im Niveau der sozialen Absicherung durchaus mit dem des Staatssozialismus messen konnte. 

Sowohl Reformeliten als auch Bevölkerungen irrten sich allerdings insofern, als 1989 dieser „soziale Kapitalismus“ oder „Teilhabekapitalismus“ (Busch, Land 2013) im Westen bereits im Verschwinden begriffen war. Sie strebten also einem westlichen System zu, das es in der Realität bereits so nicht mehr gab. Der sozialstaatlich regulierte Kapitalismus vollzog ab Mitte der 1970er Jahre einen Systemwechsel hin zum Wirtschafts- und Gesellschaftsmodell eines deregulierten Kapitalismus. In den Hauptländern des Kapitalismus zog sich der Staat in dieser Periode unter dem Eindruck der neoliberalen Meistererzählung aus der politischen Gestaltung der Wirtschaft zurück. Der osteuropäische Systemwechsel beschleunigte diesen Modellwechsel zusätzlich. 

Die Transformation erfolgte unter den Losungen von der Diktatur zur Demokratie, vom Plan zum Markt, der „Rückkehr nach Europa“. Besonders radikal war die wirtschaftliche Umgestaltung: Die zwar schlecht funktionierende, aber etablierte Arbeitsteilung zwischen den Staaten des sowjetischen Osteuropas wurde in kurzer Zeit aufgehoben und die Unternehmen mussten sich nunmehr auf dem globalisierten kapitalistischen Markt behaupten. In Osteuropa wurde die größte Privatisierungskampagne der Geschichte realisiert. Die Art der Transformation bewirkte in vielen Ländern eine Superinflation sowie eine – in diesen Gesellschaften nach Jahrzehnten der Beschäftigungsgarantie erstmals auftretende – steil ansteigende Arbeitslosigkeit. In Polen betrug sie zwischen 1995 und 2005 um 15 Prozent, teils lag sie noch höher. In der Slowakei und Bulgarien war sie jahrelang ebenso hoch. Die deutliche Kluft im Lohngefälle gegenüber dem Westen befeuerte zudem die Arbeitsmigration von Ost nach West. In einigen Ländern Südosteuropas verringerte sich die Bevölkerung durch Auswanderung und sinkende Geburtenraten um bis zu 10 Prozent. In der Ukraine seit 1990 sogar um ca. 20 Prozent. Die sozialen Auswirkungen auf die Mehrheit der Bevölkerung waren umfassend und erheblich: Sparguthaben wurden vernichtet, die Reallöhne sanken über einige Jahre deutlich. Eine größere Gruppe der Beschäftigten blieb durch Scheinselbständigkeit ohne sozialen Schutz.\endnote{Zu diesen Ergebnissen genauer meine Publikationen (Segert 2013, Kapitel 7; Segert 2015, 2018).}  

Den Härten der Bevölkerung lag eine tiefe Rezession zugrunde, die länger dauerte, als die meisten Beobachter 1989 angenommen hatten. Sie resultierte aus dem Zerfall des bisher einheitlichen Wirtschaftsraumes RGW, aus dem Prozess der Integration in die kapitalistische Weltwirtschaft und der Überwindung von inneren wirtschaftlichen Ungleichgewichten, war aber auch dadurch bedingt, dass die neoliberale Ideologie suggerierte, der Staat müsse nur loslassen, dann würde der Markt es schon richten. Besonders deutlich werden die verpassten Möglichkeiten, wenn man die wirtschaftlichen Transformationen der osteuropäischen Staaten mit denen Chinas vergleicht, wo der Staat eine deutlich größere steuernde Funktion beibehielt (Frank, Segert 2008; Land 2020). 

    \noindent Hinzu kommt die Bewertung der politischen Führungen durch die unter den Härten der Transformation leidende Bevölkerung. Sie wurden nicht nur für deren Resultate verantwortlich gemacht, sondern es wurde auch kritisch bemerkt, dass sich Teile der Elite in der stattfindenden riesigen Umverteilung des nationalen Reichtums persönlich bereicherten. Diese Art der Korruption machten sich dann andere Teile der politischen Klasse zunutze und richteten ihren Wahlkampf entsprechend gegen die etablierten Kräfte. Sowohl in den Wahlen 2001 in Bulgarien, 2005 in Polen als auch denen von 2010 in Ungarn spielte die Opposition diese Karte gegen die jeweilige Regierung aus. Und sie bekamen dafür die Unterstützung durch die frustrierte Bevölkerung. Im Ergebnis der objektiven Lage und deren Interpretation in den Wahlkämpfen verbreitete sich in der Region die Auffassung, dass Politik durch Korruption gekennzeichnet ist. In einer repräsentativen Umfrage in acht osteuropäischen Ländern und Ostdeutschland 20 Jahre nach dem politischen Umbruch waren die Befragten in sechs dieser Staaten der Meinung, dass die Korruption das wichtigste gesellschaftliche Problem sei. \endnote{Die Umfrage wurde durch das renommierte Pew Research Center im Herbst 2009 durchgeführt. Nur in Russland und der Slowakei wurde die Korruption als zweitwichtigstes Problem genannt. In Ostdeutschland gehörte sie dagegen nicht zu den drei wichtigsten Problemen. Siehe dazu den Report „End of Communism Cheered but Now with More Reservations”. URL: https://www.pewresearch.org/global/2009/11/02/end-of-communism-cheered-but-now-with-more-reservations/.}

    \section{Vertrauenskrise der Demokratie und Aufstieg der rechten Populisten}
    \noindent In den Staaten Ostmitteleuropas verloren nicht nur die Sozialdemokratie, sondern auch die Mitte-Rechts-Parteien, die im Wechsel mit ihnen seit 1989 regiert hatten. Die repräsentative Demokratie geriet insgesamt in eine tiefe Vertrauenskrise. Diese äußerte sich u. a. in einer deutlich sinkenden Wahlbeteiligung in fast allen Staaten der Region mit Ausnahme Polens, wo von Anfang an eine niedrige Wahlbeteiligung zu verzeichnen war. In einigen Ländern sank die Wahlbeteiligung auch in den jüngsten Wahlen: In Bulgarien nahmen an der letzten von drei Wahlen im Jahr 2021 nur noch 39 Prozent der Wahlberechtigten teil. In Rumänien betrug die Wahlbeteiligung bei den jüngsten Parlamentswahlen 2020 nur noch 33 Prozent.

    \begin{table*}
      \centering
      \caption{Transformationsrezession in Osteuropa nach 1989 (bis zum Beginn der Krise 2008) }
      \resizebox{\linewidth}{!}{%
      \begin{tabular}{lllll}
      \textbf{Länder} & \begin{tabular}[c]{@{}l@{}}\textbf{Erstes Jahr }\\\textbf{des BSP-}\\\textbf{Einbruchs}\end{tabular} & \begin{tabular}[c]{@{}l@{}}\textbf{Jahre des }\\\textbf{Rückgangs}\end{tabular} & \begin{tabular}[c]{@{}l@{}}\textbf{Jahre des }\\\textbf{deutlichsten }\\\textbf{Einbruchs} \\\textbf{Prozent/} \\\textbf{Jahr}\end{tabular} & \begin{tabular}[c]{@{}l@{}}\textbf{Jahr, in dem }\\\textbf{das BSP/Kopf}\\\textbf{in 1989 wieder }\\\textbf{erreicht wurde}\end{tabular} \\
      Albanien & 1990 & 1990–92, 1997 & -28 (1991) & 1999/2000 \\
      \begin{tabular}[c]{@{}l@{}}Bosnien und \\Herzegowina\end{tabular} & k. A. (Krieg) & k.A. & \begin{tabular}[c]{@{}l@{}}Erstes Jahr des \\Wieder-\\anstiegs: 1996\end{tabular} & 2008 erst 85\% \\
      Bulgarien & 1990 & 1990–93, 1996/97 & -11,7 (1991) & 2006 \\
      Tschechien & 1990 & 1990–92,
        1997/98 & -11,6 (1991) & 2000 \\
      Ungarn & 1990 & 1990–93 & -11,9 (1991) & 1999/2000 \\
      Kroatien & 1989 & 1989–93
        (Krieg); 1999 & -21,1
        (1991) & 2005 \\
      Polen & 1990 & 1990–91 & -11,6 (1990) & 1995/96 \\
      Rumänien & 1989 & 1998–92,
        1997-99 & -12,9 (1991) & 2004 \\
      Slowakei & 1990 & 1990–93 & -15,9 (1991) & 1999 \\
      Slowenien & 1989 & 1989–91 & ~ -8,9 (1991) & 1997 \\
      Serbien & 1990 & 1990–93,
        1999 (Kriege) & -30,8 (1993) & 2008 erst 73\%
      \end{tabular}
      }
      \end{table*}

    Am deutlichsten wurde diese Krise der liberalen Demokratie im Aufstieg von nationalpopulistischen Parteien ab der Jahrtausendwende. Bulgarien machte 2001 den Anfang, als eine gerade vor den Wahlen gebildete Partei aus dem Stand stärkste politische Kraft wurde. In Polen kam 2005 die PiS an die Regierung. Nach einem schnellen Machtverlust 2007 kam sie aber 2015 stabiler wieder an die Regierung. Ungarn folgte mit der Fidesz 2010, die seither in drei Wahlen ihre Mehrheit bestätigen konnte. Von Viktor Orbán stammt auch der Begriff, der die Distanz zur liberalen Demokratie am deutlichsten formuliert, der einer „illiberalen Demokratie“. Das erfolgte 2014 in einer Rede auf einer Sommeruniversität in Rumänien.  Auch in weiteren Ländern waren nationale populistische Parteien erfolgreich, etwa in der Slowakei (smer-sd).

    In Osteuropa kam es nach den wilden 1990er Jahren zu periodischen Protesten. Zentraler Grund: Korruption, aber auch ein Kampf in den Städten für mehr Teilhabe. Es gab in einigen Ländern auch bedeutsame gewerkschaftliche Kämpfe: In Tschechien für höhere Löhne im öffentlichen Bildungs- und Gesundheitswesen, in Ungarn 2018 Proteste gegen ein Gesetz, das Überstunden zugunsten der Unternehmer regelte, oder in Rumänien gegen die chronischen Defizite im Gesundheitssystems. 

    Die meisten Proteste richten sich jedoch gegen Korruption etablierter Politiker oder die Willkür von Entscheidungen bei der Entwicklung der Städte, ohne die Bewohner einzubinden. Getragen wurden sie häufig von der bessergestellten Stadtbevölkerung. Derartige Proteste führen in der Regel zur Stärkung des liberalen Lagers: In Rumänien entstand die USR, die Union zur Rettung Rumäniens, in Tschechien wurde die Piratenpartei in den Großstädten stark, in der Slowakei setzte sich mit Zuzana Čaputová eine liberale Politikerin als Präsidentin durch. Die Sozialdemokraten waren in Rumänien und der Slowakei gerade das Ziel jener Proteste. Sie wurden als Teil einer korrupten Klasse von etablierten Politikern angesehen. In Bulgarien führten die Proteste zu einem Ansehensverlust der lange regierenden bürgerlichen Partei GERB, noch mehr aber verlor die sozialdemokratische Partei BSP an Unterstützung. Populistische Parteien tauchen in diesem Land auch nach 2001 immer wieder auf. Die nach der dritten Wahl 2021 von einer neuen Partei gebildete Regierungsmehrheit ist ebenfalls nicht stabil geblieben, sondern im Juni 2022 wieder auseinandergebrochen. 
    
    Was ist eigentlich links unter heutigen Bedingungen?
    Rechte Populisten verfolgen in einigen Fällen (besonders stark die polnische Partei „Recht und Gerechtigkeit“/PiS) eine scheinbar linke sozialpolitische Agenda (Segert 2019). „Scheinbar“ meint, dass das hohe Kindergeld, das von der PiS-Regierung eingeführt wurde, seitens der Partei als eine Förderung der traditionellen Familienwerte gedacht wurde. Natürlich fällt auf, dass die vorherigen sozialdemokratischen (wie auch liberalen) Regierungen eine solche Förderung von bedürftigen Familien nicht für nötig erachteten. 

    Auf der anderen Seite wird einigen linken Parteien, auch der deutschen Partei „Die Linke“, aktuell von internen Kritikern vorgeworfen, sich nicht in ausreichendem Maße um die Lebensbedürfnisse der weniger Privilegierten zu kümmern (Harteveld 2016; Wagenknecht 2021; Balhorn 2022; Baron 2022). Damit entsteht die Aufgabe, genauer zu bestimmen, was „politisch links“ heute bedeutet.  

    Historisch gesehen war die Sache klar: „Links“ bedeutete im Lager der lohnabhängig Beschäftigten, der Arbeiterklasse zu stehen, gegen Ausbeutung und koloniale Unterdrückung zu kämpfen, für die Gleichberechtigung der Frau, für gleiche Rechte aller Menschen unabhängig von der Geburt, für gesellschaftlichen Fortschritt. Über den Weg zu diesen Veränderungen gab es verschiedene Vorstellungen zwischen kommunistischer, anarchistischer und sozialdemokratischer Linker. „Rechts“ war dagegen eine politische Position, die für die Erhaltung der bestehenden sozialen Ordnung eintrat, für Familie, Vaterland und Eigentum. Die Sicherung der individuellen Freiheit gegenüber dem Staat, gegenüber politischer Willkür, war eine wichtige Forderung einer liberalen Rechten.

    Mit der Entstehung eines sozial regulierten Kapitalismus ausgehend vom New Deal in den USA unter Präsident Franklin D. Roosevelt, der sich in Westeuropa nach 1945 etablierte (gefördert auch von der Herausforderung durch den sowjetischen Sozialismus, zumindest der Herausforderung durch dessen propagierte Ziele) verschwammen jedoch diese klaren Unterscheidungen. Nach 1968 sprach man von einer Neuen Linken, die die ursprünglichen Emanzipationsziele jetzt differenzierter und ohne revolutionäre, also gewaltsame, Methoden verfolgte (Eley 2002). Nach 1989 schien es dann keinen Bedarf mehr an einer radikal linken Position zu geben. Das hat sich allerdings mit der Krise des Modells eines deregulierten Kapitalismus wieder geändert. Ab 2008 trat die Krise des neoliberalen Kapitalismus ebenso zutage wie die Krise des sozialdemokratischen Versuchs, sich an ihn anzupassen (Lošonc, Josifidis 2021: 31).

    Was also kann heute als „linke Politik“ betrachtet werden? Ich gehe dabei von einem Beitrag des serbischen Philosophen Jovo Bakić auf der Konferenz des oben genannten Instituts für Sozialwissenschaften aus. Er stützt sich auf den italienischen Philosophen Norberto Bobbio (Bakić 2021: 240). Bobbio hatte bestritten, dass die Unterscheidung von Links und Rechts in der Politik ihre Bedeutung verloren hat. Linke Politik tritt für soziale Gleichheit ein, während die politische Rechte die Naturgegebenheit sozialer Hierarchien annimmt. Soziale Gleichheit meint dabei nicht völlige Angleichung der sozialen Bedingungen, oder gar eine bestimmte gleiche Form der Zuteilung von Lebensmöglichkeiten an alle Individuen.\endnote{Zu diesem Thema habe ich in meinem Buch „Transformation und politische Linke“ geschrieben: „Die wichtigste Orientierung linker Wertüberzeugungen ist jeder einzelne Mensch und der Schutz seiner natürlichen Lebenswelt. Die linke Forderung nach ‚Gerechtigkeit’ meint vor allem: Nicht der Zufall der Geburt soll entscheiden, sondern jeder Mensch soll die Chance auf ein gutes Leben haben.“ (Segert 2019: 10).}  Jedoch müssten Linke immer gegen die Wirtschaftsprinzipien des Kapitalismus auftreten, weil diese in ihrer Anwendung zu einer Vertiefung der sozialen Ungleichheit führten. Jovo Bakić behauptet auch, dass die Rechte der Vergangenheit und Gegenwart affirmativ entgegentrete, während die Linke auf Veränderung dränge. „Die Linke sucht nach Alternativen zu der nicht rationalen Wirklichkeit, während die Rechte die Fortsetzung der heiligen Traditionen anstrebt bzw. die empirisch bestätigte gute Praxis bekräftigt.“ (Ebd.: 241 – Übers.: D.S.) 

    Die Besonderheit der heutigen Situation wird so natürlich noch nicht deutlich. Dieser Bestimmung nähert man sich, wenn man die heutigen relevanten Strömungen der Linken ansieht. Für Serbien sieht Bakić die Situation folgendermaßen: Es gibt ein deutlich negatives Erbe des Versuchs einer Synthese von linker Symbolik und nationalistischer Politik unter Slobodan Milošević. Positiv für die Linke ist aus seiner Sicht eine Synthese von jüngeren Aktivisten, die er als „radikal demokratisch und links orientiert betrachtet“; sie äußert sich in Protestbewegungen und kleineren politischen Parteien wie der „Partija radikalne levice“ (Radikale Linkspartei). Außerdem nimmt er eine interessante Differenzierung der heutigen Linken vor: zwischen den Anarchisten, die jegliche Autorität ablehnen, den extremen Linken, die revolutionäre Gewalt für die Beseitigung des Kapitalismus und jeglicher Arbeitsteilung für nötig erachten, den radikalen Linken, für die eine Revolution nur für den Fall als erforderlich angesehen wird, in denen ein Wahlsieg der Linken durch die Wirtschaftseliten oder bisherige politische Machthaber nicht akzeptiert wird, und schließlich den moderaten Linken, den Sozialdemokraten (ebd.). Kurz gesagt, er unterscheidet verschiedene Gruppen Linker wesentlich danach, welche politischen Mittel zur Veränderung der Gesellschaft sie akzeptieren. 

    Zwei Momente erschweren eine valide Definition aktueller linker Politik. Die von Marx angenommene sich zuspitzende Konfrontation zwischen wirtschaftlichen Eigentümern, der Bourgeoisie, und den Lohnarbeitern lässt sich auf die heutige Situation nicht übertragen. Der aktuelle Kapitalismus ist durch Jahrzehnte der Sozialstaatlichkeit und Konsumgesellschaft wie auch durch Veränderungen in der Struktur der Lohnarbeiterschaft deutlich gewandelt. In dem schon zitierten Konferenzbeitrag von Alpar Lošonc und Kosta Josifidis wird auf eine zentrale Veränderung hingewiesen, deren Nichtberücksichtigung jede linke Politik schwächt: Die Präferenzen, Bedürfnisse und Interessen der Bevölkerung kapitalistischer Gesellschaften der Gegenwart sind durch die dominierende Wirtschafts- und Sozialordnung tief geprägt worden. Die radikale Linke, so die beiden Autoren, die eine Negation des Kapitalismus anstrebt, „riskiert, den Weg zu unterschätzen, auf dem die Präferenzen und Orientierungen des Volkes durch die Mechanismen des Kapitalismus geformt werden“ (Lošonc, Josifidis 2021: 31 – Übers.: D.S.). Eine Änderung der dominanten Ordnung ist – anders gesagt – nicht möglich ohne Änderung der Präferenzen von Mehrheiten der Bevölkerung und ihrer vorherrschenden „Lebensweise”.\endnote{Das ist ein Moment, auf das besonders Brand und Wissen (2017) in ihrem Buch über die „imperiale Lebensweise“ hingewiesen haben. Daneben geht es ihnen um eine Kritik der globalen Ungleichheitsverhältnisse, in denen aber auch die Mehrheitsbevölkerung der Länder des globalen „Nordens“ der des globalen „Südens“ gegenübersteht.}

    An diesem Punkt zeigen sich auch Parallelen von sozialer und ökologischer linker Politik. Auch die Transformation des Verhältnisses von Gesellschaft und Natur ist unmöglich ohne Veränderung der individuellen Verhaltensweisen einer Mehrheit der Menschen. Der Weg dahin führt allerdings nicht zuerst über moralische Appelle, sondern verlangt nach strukturellen Veränderungen von Produktions- und Lebensweise, er ist nur durch politische Eingriffe erreichbar, die ihrerseits durch individuelle Entscheidungen vorangetrieben werden. 
    
\section{Aufgaben linker Politik in Osteuropa heute}

\noindent Zahlen können ein Problem nicht vollständig erklären, aber sie können seine Dringlichkeit erläutern. Deshalb sollen hier zunächst einige Daten angeführt werden. An ihnen wird die anhaltende Aktualität des Themas soziale Gleichheit sichtbar. 

Eine wichtige Zahl unter dem Gesichtspunkt der sozialen Gerechtigkeit ist der Zusammenhang zwischen Einkommen und Lebenserwartung. „The Poor Die Young“ titelt eine Studie der FES von Michael Dauderstädt (2019), Arme sterben früher. Für einen Ost-West-Vergleich in der EU zu diesem Gegenstand verweist der Autor auf viele internationale Studien, u. a. eine des Instituts für Gesundheitsökonomie und Klinische Epidemiologie (IGKE)\endnote{Siehe „Zum Zusammenhang zwischen Einkommen und Lebenserwartung“: https://www.sozialpolitik-aktuell.de/files/sozialpolitik-aktuell/\_Kontrovers/Rente67/Zusammenhang-Einkommen-Lebenserwartung.pdf.},  das bis 2005 Karl Lauterbach leitete. In ihr findet sich Tabelle 4, die aus repräsentativen Daten in Deutschland am Beginn des Jahrtausends folgt.

\begin{table*}
  \centering
  \caption{Entwicklung der Wahlbeteiligung in einigen Staaten Osteuropas seit 1989 (in Prozent der Wahlberechtigten)}
  \resizebox{\linewidth}{!}{%
  \begin{tabular}{>{\hspace{0pt}}m{0.154\linewidth}>{\hspace{0pt}}m{0.229\linewidth}>{\hspace{0pt}}m{0.262\linewidth}>{\hspace{0pt}}m{0.29\linewidth}}
  \textbf{Land} & \textbf{Wahlbeteiligung }\par{}\textbf{bei der ersten }\par{}\textbf{Wahl nach 1989}\par{}\textbf{~(Jahr)} & \textbf{Durchschnittliche }\par{}\textbf{Wahlbeteiligung in }\par{}\textbf{den 1990er Jahren} & \textbf{Durchschnittliche }\par{}\textbf{Wahlbeteiligung }\par{}\textbf{2000–2015 (\textit{Differenz}}\par{}\textbf{\textit{zu den 1990ern)}} \\
  Tschechien & 96,3 (1990) & 83 & 61 (\textit{- 22}) \\
  Slowakei & 95,4 (1990) & 85 & 60 \textit{(-15)} \\
  Polen & 43 (1991) & 48 & 48\textbf{~~ }\textit{(0)} \\
  Ungarn & 65,1 (1990) & 63 & 61 \textit{(-2)} \\
  Rumänien & 86,2 (1990) & 80 & 59 \textit{(-21)} \\
  Bulgarien & 90,8 (1990) & 77 & 57 \textit{(-20)} \\
  Slowenien & 85,9 (1992) & 80 & 62 \textit{(-18)} \\
  Kroatien & 75,6 (1992) & 72 & 64 \textit{(-8)}
  \end{tabular}
  }
\end{table*}
   
Dauderstädt ergänzt einen Ost-West-Vergleich. In den „alten“ EU-Mitgliedsländern liegt die durchschnittliche Lebenserwartung bei 81 Jahren und mehr (mit zwei Ausnahmen: Dänemark, wo die Menschen etwas kürzer leben, und Slowenien, das im höheren westeuropäischen Durchschnitt liegt). In den neuen Mitgliedsländern sieht es folgendermaßen aus: Bis zu drei Jahre früher sterben Menschen in Polen, Tschechien, der Slowakei, Estland, Kroatien und Rumänien. Drei bis sechs Jahre früher sterben die Menschen in Bulgarien, Lettland, Litauen und Ungarn.\endnote{Ein genauerer Vergleich würde erfordern, die Entwicklung der durchschnittlichen Lebenserwartung im Staatssozialismus und nach 1989, im Transformationsprozess, einzubeziehen. Das würde diesen Beitrag allerdings überfordern. Für einen Vergleich zwischen OECD-Staaten 1960 und 2007 siehe den OECD-Bericht 2009: Gesundheit auf einen Blick, S. 15, URL: https://www.oecd.org/berlin/45364683.pdf.}

Soziale Gerechtigkeit und das Niveau der Einkommensverteilung in einem Land stehen in einer direkten Beziehung zueinander. In den skandinavischen Ländern mit einer starken sozialstaatlichen Tradition liegt die Ungleichheit in der Einkommensverteilung, abgebildet durch den Gini-Koeffizienten, niedriger als im europäischen Durchschnitt. Für Osteuropa kann keine Durchschnittsgröße gebildet werden, da die Ungleichheit in den verschiedenen Ländern zu weit auseinanderfällt: Slowenien, Slowakei, Tschechien und Kroatien liegen unter dem Durchschnitt (geringere Einkommensungleichheit), Polen, Estland und Ungarn liegen etwa im Durchschnitt der EU, Litauen, Lettland und Rumänien liegen deutlich über dem Durchschnitt und Bulgarien ist mit einem Gini-Koeffizienten von 40,3 der EU-Spitzenreiter.\endnote{Angaben nach den Daten der Weltbank für 2019: url{https://data.worldbank.org/indicator/SI.POV.GINI?view=map\&locations=EU (25.7.22).}} 

Ein weiterer Indikator der sozialen Lage ist die Einkommensarmut. Als einkommensarm wird betrachtet, wer weniger als 60 Prozent des Durchschnittseinkommens der Bevölkerung seines Landes zur Verfügung hat.\endnote{Die Armutsgefährdungsquote ist ein Indikator zur Messung relativer Einkommensarmut und wird – entsprechend dem EU-Standard – definiert als der Anteil der Personen, deren Äquivalenzeinkommen weniger als 60 Prozent des Medians der Äquivalenzeinkommen der Bevölkerung (in Privathaushalten) beträgt. (zitiert nach Wikipedia: Art. „Armutsgefährdung“ URL: https://de.wikipedia.org/wiki/Armutsgefährdung\#Definitionen\_in\_der\_Europäischen\_Union.}  Dieser Personenkreis gilt als „armutsgefährdet“. Durch sozialstaatliche Maßnahmen kann auf diese Quote reduzierend eingewirkt werden. Für 2016 sieht die Lage nach Ländern Osteuropas folgendermaßen aus: Der EU-Durchschnitt der Armutsgefährdung lag bei 17 Prozent, in Tschechien, der Slowakei, Slowenien sowie in Ungarn waren weniger Menschen als im EU-Durchschnitt armutsgefährdet, Polen lag im Durchschnitt. Die folgenden Länder Osteuropas lagen darüber: Kroatien (19,5 Prozent), die drei baltischen Staaten (bei 22 Prozent), Bulgarien (23 Prozent) und in Rumänien war ein Viertel der Bevölkerung armutsgefährdet. Schlechter war in Südosteuropa nur noch Serbien mit 26 Prozent armutsgefährdeter Bevölkerung.  Aber die Armut ist in reichen Ländern objektiv etwas anderes als in armen Ländern. Für den direkten Ländervergleich eignen sich solche relativen Größen nicht. Höhere Einkommen sind eine der Größen, die die Migration aus den ärmeren Ländern in die reicheren antreiben. 

Für gleiche Lebenschancen ist auch der Zustand des Bildungswesens und des Gesundheitswesens von zentraler Bedeutung. Die Abhängigkeit der Lebenserwartung vom Einkommen kann durch soziale Transfers und den Ausbau der sozialen Infrastruktur erheblich beeinflusst werden. Allerdings kumulieren in den (einkommensarmen) Ländern Osteuropas verschiedene negative Bedingungen. Um nur ein Beispiel zu nennen: Die Arbeitsmigration von Beschäftigten des Gesundheitswesens in Rumänien, Albanien oder anderen osteuropäischen Ländern in die reicheren Länder Westeuropas verschlechtert die bereits schlechte Lage in den ärmeren Ländern zusätzlich.\endnote{Das zeigte sich besonders deutlich zu Beginn der Corona-Pandemie, siehe hierzu Segert 2020: 4-5.}

Damit wird auch deutlich, dass eine wirksame Politik für eine gerechtere Gesellschaft nicht im nationalstaatlichen Rahmen allein durchgekämpft werden kann. Es sind zumindest EU-weite Maßnahmen wichtig. Die EU muss sozialer werden, die Überwindung der Defizite des einen Mitgliedslandes darf nicht zu Lasten anderer Mitgliedsländer gesucht werden. Am größten ist hier die Verantwortung der reicheren Staaten, die bisher von den Lohndifferenzen am meisten profitieren. Auch das ist eine Aufgabe linker Parteien, die dafür allerdings zumindest europaweit wirken müssen. Wie Irena Ristic im Vorwort zum Belgrader Konferenzband über die Krise der politischen Linken schrieb: „Ohne eine vereinigte, transnationale Front der Solidarität linker Bewegungen überall in Europa, lässt sich kaum ein nachhaltiger Einfluss auf und eine paradigmatische Veränderung des dominierenden Wirtschaftsmodells erreichen“ (2021: 15 – Übers.: D.S.). Für die Bewältigung der sozial-ökologischen Transformation in eine post-kapitalistische Gesellschaft ist allerdings eine weltweite Kooperation linker Parteien und sozialer Bewegungen nötig, weit über das hinausgehend, was bisher erreicht wurde.  

\section{Eine umfassende Demokratisierung der Demokratie}

\noindent Die Krise des Finanzsystems brachte eine Schwächung der Demokratie mit sich, die sich nicht zuletzt in einem geringeren Vertrauen in die Institutionen der repräsentativen Demokratie äußerte. Der Aufstieg national- und sozialpopulistischer Parteien und ihre Regierungsübernahme ist ein deutliches Merkmal dieses geschwundenen Vertrauens. Die politische Linke hingegen konnte von dieser Krise bisher nicht profitieren. Im zitierten Belgrader Sammelband (Ristic 2021) wird die These aufgestellt, dass die linken Parteien in die Krisen des Kapitalismus eingeflochten seien. Das gelte sowohl für die moderate als auch für die radikale Linke. Für letztere wäre besonders charakteristisch, dass ihr die Einbindung der Präferenzen ihrer potentiellen Anhängerschaft in die kapitalistische Reproduktion zu wenig bewusst sei (Lošonc, Josifidis 2021: 31). Wenn diese These richtig ist, dann bedarf es für eine Überwindung der Krise der politischen Linksparteien einer Stärkung der politischen Autonomie ihrer potentiellen linken politischen Basis. 

In Osteuropa steht einem sich über die letzten Jahre belebenden linken Diskurs, dem offensiven Auftreten linker Intellektueller (Slavoj Žižek, Irena Ristic, Boris Kagarlitsky, Béla Greskovits) eine mangelnde Unterstützung linker Parteien in den Bevölkerungen gegenüber. Der Niedergang der gewandelten postkommunistischen Sozialdemokraten wurde oben dargestellt. Es sind einige neue linke Parteien entstanden, die allerdings bestenfalls in den Städten auf eine relevante Anhängerschaft zurückgreifen können, wie in Zagreb oder Ljubljana. Landesweit sind sie, falls überhaupt in den Parlamenten vertreten, politisch marginalisiert. In Polen hat Razem (heute: Lewica Razem) eine gewisse Bedeutung, die sie innerhalb eines Wahlbündnisses mit der SLD erreichte. Sie stellt sechs von 460 Abgeordneten des Sejms. In Kroatien erreichte ein linkes Wahlbündnis dreier Parteien (Možemo, Radnicka Fronta, Nova Levjica) sieben von 151 Sitzen des Sabor. In Nordmazedonien ist eine linke Partei, Levica, mit zwei von 120 Sitzen im Parlament. In Ungarn hat die linksliberale Partei Párbeszed (Párbeszéd Magyarországért) auf einer gemeinsamen Liste mit den Sozialdemokratien sieben Sitze von 199. Das sind Ergebnisse, die – selbst gemeinsam mit den Resultaten der Sozialdemokraten in den betreffenden Ländern – weit entfernt sind von dem Ziel, eine Massenpartei der arbeitenden Bevölkerung zu werden, an das kürzlich Loren Balhorn im „Jacobin“ (2022) vor dem Hintergrund der Wahlniederlage der Partei „Die Linke“ erinnert hat.

Den linken Parteien in Osteuropa ist offenbar der Mut abhandengekommen, die ganze Gesellschaft zu verändern. Ihnen ist jedenfalls ein übergreifendes Ziel verloren gegangen. Dieses Ziel muss heute nicht in einem einzigen Begriff zusammengefasst werden, wie das bei den radikalen Linken vor 1989 der „Sozialismus“ (als demokratischer oder ökologischer oder freiheitlicher) war. Es kann auch eine Vielzahl von zusammenhängenden Aufgaben umfassen, welche auf verschiedenen Wegen und nur auf Grundlage diverser Bündnisse politischer Akteure verwirklicht werden können. Am Ende werden Gesellschaft und Lebensweise eine andere sein. 

Eines muss allerdings auf jeden Fall geschehen: eine Aktivierung größerer Bevölkerungsgruppen für die eigenen Interessen. Sie müssen sich von der Haltung eines dankbaren Klienten gegenüber dem Patron befreien. Dafür ist die umfassende Demokratisierung der in der Mehrzahl der osteuropäischen Gesellschaften gegebenen demokratischen Strukturen notwendig. Die Interessen der weniger besitzenden und sozial schwächeren sozialen Gruppen können nur durch das Zusammenspiel von politischen Parteien in Parlament und Regierung und selbstbewussten Aktivitäten der entsprechenden sozialen Schichten selbst an den Wahlurnen und in Protesten durchgesetzt werden. Die Schwäche der linken Parteien in der Region ist oben dargestellt worden. Die Stärke der Initiative der benachteiligten Menschen selbst hängt davon ab, ob sie eine ausreichende Kraft und ein für politisches Handeln nötiges Selbstbewusstsein entwickeln können. 

Der Einsatz linker politischen Parteien für eine stärkere soziale Absicherung der arbeitenden Bevölkerung, für den Wiederaufbau eines funktionsfähigen Sozialstaates, hat dabei eine wichtige unterstützende Bedeutung. Sie wird auch für die Wiedererlangung politischer Kraft der linken Parteien entscheidend sein. Vor allem aber ist für eine nachhaltige Demokratie, die sich auf eine aktive und autonom entscheidende Mehrheit der Bevölkerung stützt, ein bestimmtes Niveau der sozialen Absicherung der Lohnabhängigen eine wichtige Voraussetzung. Die von den Anforderungen kapitalistischer Erwerbsarbeit herausgeforderten Beschäftigten können anders nur schwer die Zeit aufbringen, die nötig ist, um sich in der eigenständigen Sphäre der Politik zurechtzufinden. 

Eine weitere Aufgabe linker Politik ist die strukturelle Öffnung der Demokratie. Die parlamentarische Demokratie muss durch die Ausweitung bestehender und neue Formen gesellschaftlicher Demokratie in Betrieben und Institutionen ergänzt werden. Aus einer Wahldemokratie mit abgesicherten politischen Rechten muss eine partizipative Demokratie werden. Und die entsprechende Entwicklung in Osteuropa ist ohne eine Wandlung in den reicheren, westlichen Gesellschaften in eine ähnliche Richtung unmöglich.\par

\printendnotes[custom]

\section{Literatur}
    \begin{bibdescription}
        \item Bakić, Jovo (2021): What’s left of the Left in Serbia following the restoration of capitalism? In: Ristic, Irena (Hrsg.): Resetting the Left in Europe. Challenges, Attempts and Obstacles. Belgrade: IfSS, S. 239-257.
        \item Balhorn, Loren (2022): Mosaik, Steine, Scherben. In: Jacobin online. URL: https://jacobin.de/artikel/mosaik-steine-scherben-die-linke-linkspartei-mosaiklinke-bundesparteitag/.
        \item Baron, Christian (2022): Wo bleibt die Kapitalismuskritik? In: Süddeutsche Zeitung, 17.7.2022. URL: https://www.sueddeutsche.de/kultur/armut-klasse-linke-politik-identitaetspolitik-1.5621771?reduced=true.
        \item Boyer, Christoph (2008): Zwischen Pfadabhängigkeit und Zäsur: Ost- und westeuropäische Sozialstaaten seit den siebziger Jahren des 20. Jahrhunderts. In: Jarausch, Konrad H. (Hrsg.): Das Ende der Zuversicht? Die siebziger Jahre als Geschichte. Göttingen: Vandenhoeck \& Ruprecht, S. 103-119.
        \item Bozóki, András; Ishiyama, John T. (Hrsg.) (2002): The Communist Successor Parties of Central and Eastern Europe. Armonk, N. Y.: M. E. Sharpe.
        \item Brand, Ulrich; Wissen, Markus (2017): Imperiale Lebensweise: Zur Ausbeutung von Mensch und Natur im globalen Kapitalismus. München: oekom verlag.
        \item Busch, Ulrich; Land, Rainer (2013): Teilhabekapitalismus – Aufstieg und Niedergang eines Regimes wirtschaftlicher Entwicklung am Fall Deutschland 1950 bis 2010. Ein Arbeitsbuch. Norderstedt: Books on Demand.
        \item Clement, Hermann u. a. (2002): Wachstum in schwierigem Umfeld: Wirtschaftslage und Reformprozesse in Ostmittel- und Südosteuropa sowie der Ukraine 2001-2002. Working Paper 242, München: OEI. URL: https://www.dokumente.ios-regensburg.de/publikationen/wp/wp242k.pdf. 
        \item Dauderstädt, Michael (2019): \#EUngleich: The Poor Die Young, FES Wirtschafts- und Sozialpolitik. URL: https://www.fes.de/abteilung-wirtschafts-und-sozialpolitik/artikelseite-wiso/eungleich-the-poor-die-young.
        \item EBRD (2009): Transition Report 2009. Transition in Crisis? URL: https://www.ebrd.com/downloads/research/transition/TR09.pdf 
        \item Eley, Geoff (2002): Forging Democracy – the history of the Left in Europe, 1850–2000. Oxford: Oxford University Press. 
        \item Frank, Rüdiger; Segert, Dieter (2007): „Postsozialismus“ in Ostasien und Osteuropa? Ziele und Grundlagen eines Vergleichs. In: Segert, Dieter (Hg.) (2007): Postsozialismus: Hinterlassenschaften des Staatssozialismus und neue Kapitalismen in Europa. Wien: Braumüller, S. 123-158.
        \item Fuchs, Susanne; Offe, Claus (2008): Welfare State Formation in the Enlarged EU. Patterns of Reform in the Post-Communist New Member States. Berlin: Hertie School of Governance. Working Papers, Nr. 14, April 2008. 
        \item Fukuyama, Francis (1992): Das Ende der Geschichte: Wo stehen wir? München: Kindler.
        \item Harteveld, Eelco (2016): Winning the ‘losers’ but losing the ‘winners’? The electoral consequences of the radical right moving to the economic left. In: Electoral Studies 44, S. 225-234.
        \item Hertle, Hans-Hermann (2006): Wunderwirtschaft – Konsumsozialismus. In: Ders.; Wolle, Stefan (Hg.): Damals in der DDR. Der Alltag im Arbeiter- und Bauernstaat. München: Bertelsmann, S. 199-297.
        \item Kornai, Janos (2006): The great transformation of Central Eastern Europa. In: Economics of Transition 14, H. 2, S. 207-214.
        \item Land, Rainer (2020): Chinas gelenkte Marktwirtschaft und die Seidenstraßen-Initiative. In: Berliner Debatte Initial 31, H. 4, S. 2-22.
        \item Lauterbach, Karl u. a. (o. J.): Zum Zusammenhang zwischen Einkommen und Lebenserwartung. Köln: Institut für Gesundheitsökonomie und Klinische Epidemiologie. URL: https://www.sozialpolitik-aktuell.de/files/sozialpolitik-aktuell/\_Kontrovers/Rente67/Zusammenhang-Einkommen-Lebenserwartung.pdf.
        \item Lošonc, Alpar; Josifidis, Kosta (2021): The Left between crises: Antinomy between powerlessness and power. In: Ristic, Irena (Hrsg.): Resetting the Left in Europe. Challenges, Attempts and Obstacles. Belgrade: IfSS, S. 26-53.
        \item Melzer, Alex (2003): 12 Jahre Ostzusammenarbeit. Band 1: Die Transition und ihr Schatten, Bern: DEZA.
        \item Parties and Elections in Europe. Homepage, erstellt von Wolfram Nordsieck. URL: http://www.parties-and-elections.eu/.
        \item Ristic, Irena (Hrsg.) (2021): Resetting the Left in Europe. Challenges, Attempts and Obstacles. Belgrade: IfSS.
        \item Segert, Dieter (2013): Transformationen in Osteuropa im 20. Jahrhundert. Wien: facultas/UTB.
        \item Segert, Dieter (2015): Societal transformations in Eastern Europe after 1989 and their preconditions. In: The Revolutions of 1989. A Handbook. Wien: Verlag der ÖAW, S. 469-489.
        \item Segert, Dieter (2018): Eastern Europe after 1989 – a laboratory for the sustainability of „Western democracy”? In: Europejskij Przegląd Prawa I Stosunkow Międzynarodowych 4, H. 1, S. 5-17.
        \item Segert, Dieter (2019): Offers for the Disadvantaged. In: Guérot, Ulrike; Hunklinger, Michael (Hrsg.): Old and New Cleavages in Polish Society. Krems: Edition Donau Universität, S. 55-61.
        \item Segert, Dieter (2020): Corona in Südosteuropa. In: WeltTrends 165, S. 4-7.
        \item Staritz, Dietrich (1996): Geschichte der DDR (Erweiterte Neuausgabe). Frankfurt a. M.: Suhrkamp.
        \item Ther, Philipp (2019): Das andere Ende der Geschichte. Über die Große Transformation. Berlin: Suhrkamp.
        \item Wagenknecht, Sahra (2021): Die Selbstgerechten: Mein Gegenprogramm – für Gemeinsinn und Zusammenhalt. Frankfurt a. M., New York: Campus.
        
    \end{bibdescription}

\end{multicols*}