\bdichapter{<AUTOREN>}{<>TITEL}{<FORMATIERTER TITEL>}{<UNTERTITEL>}

%damit werden zwei Spalten gesetzt
\begin{multicols*}{2}
    %\noindent verhindert das Einrücken des ersten Worten eines Absatzes und sollte immer vor dem ersten Absatz nach Überschriften oder Titel geschrieben werden
    \noindent

    \section{<ÜBERSCHRIFT IM LAUFENDEN TEXT>}

    %\noindent verhindert das Einrücken des ersten Worten eines Absatzes und sollte immer vor dem ersten Absatz nach Überschriften oder Titel geschrieben werden
    \noindent Jemand musste Josef K. verleumdet haben, denn ohne dass er etwas Böses getan hätte, wurde er eines Morgens verhaftet. »Wie ein Hund!« sagte er, es war, als sollte die Scham ihn überleben. Als Gregor Samsa eines Morgens aus unruhigen Träumen erwachte, fand er sich in seinem Bett zu einem ungeheueren Ungeziefer verwandelt. Und es war ihnen wie eine Bestätigung ihrer neuen Träume und guten Absichten, als am Ziele ihrer Fahrt die Tochter als erste sich erhob und ihren jungen Körper dehnte. 
    
    %Textauszeichnugen im Fließtex sind {\textit KURSIVER TEXT} und {\textbf FETTDRUCK}
    "Es ist ein eigentümlicher Apparat", sagte der Offizier zu dem {\it Forschungsreisenden} und überblickte mit einem gewissermaßen bewundernden Blick den ihm doch wohlbekannten Apparat. {\bf Sie hätten noch} ins Boot springen können, aber der Reisende hob ein schweres, geknotetes Tau vom Boden, drohte ihnen damit und hielt sie dadurch von dem Sprunge ab. In den letzten Jahrzehnten ist das Interesse an Hungerkünstlern sehr zurückgegangen. 
    
    %Endnoten werden durch \endnote{<ENDNOTENTEXT>} eingefügt
    Aber sie überwanden sich, umdrängten den Käfig und wollten sich gar nicht fortrühren.Jemand musste Josef K. verleumdet haben, denn ohne dass er etwas Böses getan hätte, wurde er eines Morgens verhaftet. "Wie ein Hund!" sagte er, es war, als sollte die Scham ihn überleben\endnote{<TEXT DER ENDNOTE>}

    %Ausgabe der Endnoten nach dem eigentlichen Artikeltext. Bitte unverändert stehen lassen
    \printendnotes[custom]

    %Literaturverzeichniseintrag \item{TEXT FÜR LIT-VERZEICHNIS}
    \section{Literatur}
        \begin{bibdescription}
            \item Franz Kafka, Der Prozess
            \item Eintrag 2
            \item Eintrag 3 
            \item etc. pp.
        \end{bibdescription}
    \end{multicols*}
\end{multicols*}