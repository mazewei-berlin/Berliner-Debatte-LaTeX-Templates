\newpage
\begin{multicols*}{2}

\bdirezension{Anne Hartmann, Reinhard Müller (Hg.)}{Tribunale als Trauma}{Wladislaw Hedeler}

\noindent Die vorliegende Dokumentation, so steht es in der am vierzigsten Tag des Krieges gegen die Ukraine geschriebenen Danksagung, ist selbst zum Dokument einer „Epoche der Möglichkeiten“ geworden, „die nun schon weit zurückzuliegen scheint und an die wohl auf längere Sicht nicht angeknüpft werden kann“ (451). Ein bitteres Fazit, nicht nur mit Blick auf die im Buch benannten offenen Fragen und Vorschläge für weiterführende Forschungen. Wohl dem, der die Zeit seines Archivaufenthaltes in der Russischen Föderation genutzt hat, um Kopien der deklassifizierten, danach jedoch z. T. wieder sekretierten Akten zu bestellen, und mit den oft in geringer Auflage (zwischen 300 und 1000 Exemplaren) publizierten Dokumenteneditionen im Gepäck nach Deutschland zurückgeflogen ist.

Das Abebben der „Archivrevolution“ ist von regelmäßig nach Russland reisenden Historikern zur Genüge beschrieben worden und muss hier nicht referiert werden. Anne Hartmann und Reinhard Müller skizzieren die Anfänge ihrer „Spurensuche“ in den hoffnungsfrohen Jahren der Perestroika, sie erwähnen die sich ständig ändernden Arbeitsbedingungen und die Auswirkungen auf den Arbeitsalltag in den Archiven, von Eugen Ruge im Prolog zum Roman „Metropol“ als kafkaeskes Erlebnis geschildert (Ruge 2019: 7-11).

Anne Hartmann, Slawistin und Germanistin, Mitarbeiterin an der Ruhr-Universität Bochum, und Reinhard Müller, zuletzt Mitarbeiter des Hamburger Instituts für Sozialforschung, beide exzellente Kenner des literarischen Exils in der Sowjetunion, haben die im Band vorgestellten und kommentierten Dokumente während ihrer über 30 Jahre währenden Recherchen in Moskauer Archiven zusammengetragen.

Am Anfang stand das von Müller 1989 im Zentralen Parteiarchiv (ZPA) in Moskau eingesehene und 1991 veröffentlichte Stenogramm einer geschlossenen Parteiversammlung. (Müller 1991). Im Auftrag der Deutschen Kommunistischen Partei nach Moskau gereist, hatte er im ZPA Kaderakten deutscher Kommunisten einsehen dürfen. Mit anderen Unterlagen zum deutschen Emigranten Karl Schmückle lag eines Tages das o. g. Stenogramm im Lesesaal des Archivs für ihn bereit.

Ein ähnlicher Zufallsfund im Staatsarchiv der Russischen Föderation (GA RF) hatte Hartmann 2005 auf die Spur von Lion Feuchtwanger gebracht. „Tribunale als Trauma“ greift Themen auf, die sie in ihrer Studie über Feuchtwangers Bericht über seinen Moskauaufenthalt 1937 als eine der großen „Zerreißproben“ des Exils untersucht hat (Hartmann 2017: 76) und Müller in „Menschenfalle Moskau“ (Müller 2001). Die Rede ist von den Mechanismen der Illusionsbildung und Verdrängung, von Fassungslosigkeit und Unverständnis, Befremdung und Entsetzen.

Dem Band ist ein „Vorwort der Herausgeber“ (9-13), eine von Hartmann verfasste Einleitung (14-28) sowie eine von Müller erstellte „(Kultur-)Politische Chronik 1933 bis 1941: Skandale der Gleichzeitigkeit“ (29-52) vorangestellt. Drei von Hartmann eingeleitete Teile „Prozessbeginn – das Schlüsseljahr 1936“ (55-138), „Provokation und Verbandsausschlüsse“ (141-331) und „Heillos verstrickt – Appell an die ‚Instanzen‘“ (335-436) enthalten insgesamt 38 zwischen Mai 1936 und Februar 1941 entstandene Dokumente.

Aus den veröffentlichten Dokumenten – von internen Berichten und Verlagsgutachten bis hin zu Artikeln in der Tagespresse – wird deutlich, als was sich die „Kulturkader“ verstanden und wie sie fungierten. Auf dieser Folie könnten und sollten die hinsichtlich ihres Aussagewerts immer wieder kontrovers diskutierten Kaderakten (stellvertretend seien die von Hugo Huppert, Alfred Kurella und Gustav von Wangenheim genannt) gegengelesen werden. Für die Herausgeber „waren sie Wissende, (weitgehend) Eingeweihte, die mit ihren künstlerischen Fähigkeiten, aber auch den kulturpolitischen Aktivitäten in die Mechanik der Macht eingespannt waren“ (21).

Wie die Schriftsteller-Solisten in den Jahren des gewöhnlichen Stalinismus sowohl während der internen Proben hinter verschlossenen Türen als auch bei öffentlichen Auftritten agierten, kann hier nachgelesen werden. Der Informationsfluss von unten nach oben ist in seinen vielfältigen Facetten gut dokumentiert. In diesen Kontext gehört auch ein während des Prozesses gegen Nikolai Bucharin von Kurella am 8. März 1938 verfasstes Schreiben, in dem er den Organen seine Mithilfe bei der Aufdeckung der Hintergründe für die „offensichtliche Ermordung“ von Henri Barbusse anbietet und auf sein „Insiderwissen“ verweist. Nikolai Jeshow, Volkskommissar des Inneren, leitete den Brief umgehend an Stalin und Wjatscheslaw Molotow weiter (Kurella 1938: 831-832).

Es ist ebenso ernüchternd wie bedrückend zu erfahren, was die zu Wort kommenden Exilanten von Johannes R. Becher bis Hedda Zinner übereinander dachten und zu Papier brachten. Was hingegen, der Archivlage geschuldet, unterbelichtet bleibt, ist die Rolle des Dirigenten des Ganzen. Man darf auf den von Hartmann angekündigten Band, der eine breitere Darstellung des Kulturexils zum Inhalt hat (25, Anm. 34), gespannt sein.

Wie ging Stalin mit den „Ingenieuren der Seele“ um? In Russland sind nur einige wenige Publikationen und Dokumenteneditionen zum Thema „Kunst und Macht“ erschienen. Wie wurde die Linie vom Verband in den einzelnen Sektionen, darunter der deutschen, der das Buch gewidmet ist, durchgesetzt? Alexander Schtscherbakow, der den Schriftstellerverband von 1932 bis 1936 leitete, war gleichzeitig Abteilungsleiter im ZK; einmal jährlich trat er bei Stalin zum Rapport an. Wie es während der im Hause von Maxim Gorki stattgefundenen Treffen Stalins mit Schriftstellern zuging, ist dokumentiert. Alexander Tolstoi, Wladimir Stawskis rechte Hand, leitete den Verband von 1936 bis 1938 und war nie zur Berichterstattung bei Stalin im Kreml, die Sekretäre des Sowjetischen Schriftstellerverbandes Wladimir Stawski einmal (im Oktober 1937) und Alexander Fadejew – er leitete den Verband von 1938 bis 1944 – dreimal (im Januar 1939). (Na prieme 2010: 708, 720)

Es bleibt die Frage nach der „richtungsweisenden Anleitung“ des Verbandes durch die zuständige Abteilung für Kultur im ZK der KPdSU(B) bzw. der ausländischen Sektionen des Verbandes durch die Sekretariate der Komintern. Wenn diese Bestände aus den Moskauer Behördenarchiven bzw. den Staatlichen Archiven für die Forschung freigegeben sind, lassen sich diese und andere Fragen hoffentlich beantworten.\\\bigskip


\fbox{\parbox[c][][c]{0.4\textwidth}{Anne Hartmann, Reinhard Müller (Hg.): Tribunale als Trauma. Die Deutsche Sektion des Sowjetischen Schriftstellerverbands. Protokolle, Resolutionen, Briefe (1935–1941) (akte exil. neue folge, Bd. 3). Göttingen: Wallstein Verlag 2022, 469 Seiten.}}

\section{Literatur}
\begin{bibdescription}
    \item Hartmann, Anne (2017): „Ich kam, ich sah, ich werde schreiben.“ Lion Feuchtwanger in Moskau 1937. Eine Dokumentation. Göttingen: Wallstein Verlag (akte exil. neue folge, Bd. 1). 
    \item Kurella, Alfred (1938): K voprosu o smerti Anri Barbjusa. [Zur Frage nach dem Tod von Henri Barbusse.] In: Process Bucharina 1938. Dokumenty. Moskva: MFD. Sostaviteli: Ž. V. Artamonova, N. V. Petrov, S. 831-832.
    \item Müller, Reinhard (1991) (Hg.): Georg Lukács, Johannes R. Becher, Friedrich Wolf u. a. Die Säuberung. Moskau 1936: Stenogramm einer geschlossenen Parteiversammlung. Reinbek: Rowohlt.
    \item Müller, Reinhard (2001): Menschenfalle Moskau. Exil und stalinistische Verfolgung. Hamburg: Hamburger Edition.
    \item Na prieme u Stalina (2010): Tetradi (zurnaly) zapisej lic, prinjatych I. V. Stalinym (1924–1953 gg.). Moskva: Novyj chronograf.
    \item Ruge, Eugen (2019): Metropol. Roman. Hamburg: Rowohlt.
\end{bibdescription}

% rezension startet auf der nächsten Halbseite zu starten
\vfill\null\columnbreak
    \bdirezension{Sonia Combe}{Loyal um jeden Preis.\smallskip „Linientreue Dissidenten“ im Sozialismus}{Ulrich Busch}
    
    \noindent Die französische Forscherin Sonia Combe, renommierte Autorin zur Geschichte des Sozialismus in Mittel- und Osteuropa, insbesondere der DDR, hat mit vorliegendem Buch eine Arbeit veröffentlicht, die vor allem in Ostdeutschland auf großes Interesse stoßen wird. Es ist die Geschichte der Gründergeneration der DDR, genauer gesagt, eines Teils derselben, nämlich der Remigranten, die nach 1945 aus dem Exil zurückgekehrt sind und sich ganz bewusst für ein Leben in der DDR entschieden haben. Ihr Ziel war es, im Osten Deutschlands einen antifaschistischen Staat aufzubauen und eine sozialistische Gesellschaftsordnung zu schaffen. Nicht wenige von ihnen waren jüdischer Herkunft, viele Kommunisten. Alle aber waren Antifaschisten. 

    Die Autorin hat mit einer Reihe prominenter Vertreter dieses Kreises seit den 1980er Jahren intensive Gespräche geführt und ihre Bücher, Briefe, Aufzeichnungen und Erinnerungen gelesen. Dabei konnte sie feststellen, dass ihre Gesprächspartner trotz „erlebter Entzauberung“, mancher Enttäuschung und fortgesetzter Desillusionierung im realen Sozialismus mehrheitlich ihre weltanschaulichen und politischen Überzeugungen bewahrt haben. Und zwar bis zuletzt. Als Antifaschisten, Verfolgte des NS-Regimes und Kommunisten, gegenüber dem SED-Regime aber überwiegend kritisch eingestellte Intellektuelle, bildeten sie in der DDR eine Art „Schicksalsgemeinschaft“ (229). Indem Sonia Combe hier nun, mit einigem zeitlichen Abstand, ihre Geschichte erzählt, unternimmt sie den mutigen Versuch, „eine Erinnerung zu beleben, die in der postkommunistischen Geschichtsschreibung wenig präsent ist“ (12). Was dabei herausgekommen ist, ist aber keine kontrafaktische Geschichtsphantasie, wie man sie oftmals bei Zeitzeugen findet, sondern eine nützliche Ergänzung der offiziellen Dokumentation, gespeist aus individuellen Lebensläufen, kommentierten Erinnerungen und persönlichen Eindrücken. Vor allem geht es dabei, so ist im Vorwort zu lesen, „um die Geschichte derer, die geschwiegen haben, aber nicht etwa aus Angst oder Feigheit, sondern weil sie ihrem Ideal treu geblieben sind“ (12). Von dieser „Loyalität um jeden Preis“ wird in diesem Buch berichtet. 

    Der Preis der Loyalität war vor allem ein „großes Schweigen“. Worüber aber wurde in der DDR so beharrlich geschwiegen? Und was wurde dabei verschwiegen? Die Erniedrigungen und Demütigungen im „gewöhnlichen Faschismus“ waren es nicht. Auch nicht die schlimmen Erfahrungen von Flucht, Deportation, Lagerhaft und Krieg. Jeder, der in der DDR aufgewachsen ist, kennt Berichte hierüber, Bücher, Bilder, Filme, Dokumentationen und Erzählungen. Oftmals sogar aus erster Hand, von ehemaligen Spanienkämpfern, KZ-Häftlingen, Lagerinsassen oder Emigranten verfasst. Was er aber nicht kennt, das sind Berichte von Kommunisten, die nach 1933 in die UdSSR emigriert sind, die Schicksale der in den Lagern des GULAG Verschwundenen, die Geschichten von Menschen jüdischer Herkunft, die auch nach 1945 antisemitischen Anfeindungen ausgesetzt waren, die Biografien von Sozialdemokraten, die 1946 Mitglieder der SED wurden, dann aber politisch „kaltgestellt“ worden sind, und Biografien von Westemigranten, die nach dem Krieg in die DDR gekommen sind, hier einige Privilegien genossen, zeitlebens aber unter Generalverdacht standen – und trotzdem an der DDR und deren Gesellschaftsentwurf festhielten. Diese Lücke im Kenntnisstand und bei der historischen Aufarbeitung versucht das Buch von Combe zu schließen, zumindest aber ein wenig zu verkleinern. 

    Die Personen, um die es in dem Buch geht, sind zumindest älteren Lesern gut bekannt. Sie waren Künstler wie Johannes R. Becher, Bertolt Brecht, Volker Braun, Ernst Busch, Hanns Eisler, Franz Fühmann, Louis Fürnberg, Stephan Hermlin, Stefan Heym, Hermann Kant, Heiner Müller, Anna Seghers, Bodo Uhse, Helene Weigel, Wolf Biermann, Christa Wolf, Konrad Wolf und Arnold Zweig, Wissenschaftler wie Fritz Behrens, Rudolf Bahro, Ernst Bloch, Herbert Crüger, Wassili Grossman, Robert Havemann, Wolfgang Heise, Victor Klemperer, Jürgen Kuczynski, Georg Lukács, Hans Mayer, Werner Mittenzwei, Lászlo Radványi, Wolfgang Ruge, Wolfgang Steinitz und Kurt Stern. Oder Politiker wie Hermann Axen, Horst Brasch, Klaus Gysi, Kurt Hager und Markus Wolf. Auch Publizisten, Verleger usw. wie Edith Anderson, Gerhart Eisler, Walter Janka, Max Schroeder und andere. Nicht wenige von ihnen besaßen so etwas wie eine doppelte Existenz: Sie waren als Politiker, Funktionäre usw. Repräsentanten des Staates und der Partei und damit „öffentliche Personen“, als Künstler oder Wissenschaftler aber hatten sie eine kritische Distanz zu diesem Staat und zur Politik der SED, deren Mitglieder sie häufig waren. Im Falle von Becher nahm der daraus resultierende Konflikt zunehmend tragische Züge an. Aber auch andere waren keineswegs davon frei. Ausführlich erzählt wird im Buch beispielsweise über das beredte Schweigen von Anna Seghers und ihre „Kunst des Ausweichens“ (123f.) und über die taktischen politischen Manöver von Bertolt Brecht (106ff.). 

    Bemerkenswert sind einige Zuspitzungen und verallgemeinernde Wertungen. So zum Beispiel, wenn die Autorin über die „intellektuellenfeindliche Atmosphäre“ in der DDR in den 1950er Jahren berichtet und dafür als Kronzeugen sogar Kurt Hager heranzieht (125f.). Oder wenn sie sich darüber beklagt, dass die ihres Erachtens nach immer vorhanden gewesene politische Opposition innerhalb der SED bei der Aufarbeitung der DDR-Geschichte „völlig unterbewertet“ (142) bleibt. Sie verdeutlicht dies anhand der Veröffentlichung des „Spiegel-Manifests“ von 1977/78 (142), dessen Verfasser (Hermann van Berg) SED-Mitglied und aktiver Hochschullehrer an der Berliner Humboldt-Universität war. Interessant sind in diesem Kontext auch die Ausführungen zur Rezeption von Georg Lukács in der DDR (117f.). Nicht korrekt ist jedoch die Aussage, dass die Werke von Lukács nach 1956 in der DDR erst wieder in den 1980er Jahren verlegt worden seien (118). Es gab auch in den 1970er Jahren Veröffentlichungen seiner Arbeiten, so zum Beispiel 1977 im Reclam-Verlag. 

    Ein eigenes Kapitel ist dem Wirtschaftshistoriker Jürgen Kuczynski gewidmet (177-196). Auch der Untertitel des Buches wurde dem zweiten Band der Autobiografie Kuczynskis von 1992 entlehnt. J. K., wie er sich selbst gern nannte, war ein international geschätzter Historiker, aber auch ein äußerst vielseitiger Autor, Redner und Publizist. Das hatte jedoch seinen Preis. Bekannt ist seine Rolle als „Ghostdenker“ (189) für Partei- und Staatschef Erich Honecker, dem er seine Analysen vorlegte und manchmal auch unterschob, um sie danach als Beleg für die Richtigkeit und politische Korrektheit seiner eigenen Auffassung zu zitieren. In der scientific community fand ein solches Vorgehen wenig Anerkennung, aber Kuczynski war eben auch der einzige Wissenschaftler von Rang in der DDR, der sich öffentlich rühmen konnte, Honecker „zum Freund“ zu haben. Gleichwohl galt er als „der kritischste Gegner innerhalb der Partei“ (191). Dass seine kritischen Stellungnahmen ihm nie ernstlich geschadet haben, ist vermutlich auf „seine wissenschaftliche Reputation“ zurückzuführen, vielleicht aber auch auf sein „robustes Herz“ (195), mutmaßt Combe. Sonst wäre ihm vermutlich ein Schicksal wie Wolfgang Heise, der mit 61 Jahren einen tödlichen Herzinfarkt erlitten hat, beschieden gewesen. Oder wie Fritz Behrens und Wolfgang Steinitz, die ihre Maßregelung überlebt haben, aber „von der Partei gebrochen worden“ sind (195). 

    Der 2019 von Sonia Combe in Frankreich veröffentlichte Text wurde von der Romanistin und Kulturwissenschaftlerin Dorothee Röseberg sachkundig und einfühlsam ins Deutsche übersetzt. Dabei ging es ihr auch darum, den deutschen Lesern ein hierzulande bislang wenig erschlossenes und kaum bekanntes Kapitel der „DDR-Intellektuellengeschichte“ über einen „Blick von außen“ nahe zu bringen. Im Nachwort betont die Übersetzerin deshalb, dass Sonja Combe mit ihrem Buch „einen Einspruch formuliert“ hat, der sich „gegen das Vergessen jener Gründergeneration und deren Erben richtet“ (227), die, nach den Vorstellungen bundesdeutscher Geschichtsschreibung, längst „auf der Müllhalde“ der Geschichte gelandet seien. Das Buch widerspricht einer solchen Darstellung und trägt dazu bei, hier zu einer Korrektur und einer weniger einseitigen Aufarbeitung der DDR-Vergangenheit zu kommen. Im Anhang finden sich zahlreiche Anmerkungen, die für das Verständnis der Ausführungen von Bedeutung sind. Auch das angefügte Personenregister erweist sich als hilfreich bei der Erschließung des historischen Fundus, den dieses Buch den Lesern bietet. \\\bigskip
    
    \fbox{
        \parbox[c][][c]{0.4\textwidth}{
            Sonia Combe: Loyal um jeden Preis. „Linientreue Dissidenten“ im Sozialismus. Berlin: Ch. Links Verlag 2022, 268 Seiten. 
        }
    }
    
    \bdirezension{Dieter Klein}{Regulation in einer solidarischen Gesellschaft. Wie eine sozial-ökonomische Transformation gelingen könnte}{Dieter Segert}

    \noindent Manchmal liest man Bücher von ihrem Ende her. Anna Seghers beginnt das „Siebte Kreuz“, um solche Art Neugierde zu befriedigen, mit dem guten Ausgang der Geschichte. Das Ende jedenfalls zählt. Dieter Klein beendet sein neuestes Buch mit einem Abschnitt über die nötige Befreiung der Menschen aus ihrer Unmündigkeit durch die Überwindung der „von den Herrschenden gesetzten Denkschranken“ (245). Er zitiert Bourdieu: „Politische Subversion setzt kognitive Subversion voraus, Subversion der Weltsicht“ (247). Linke Politik steht vor der Aufgabe, eine Mehrheit zu mobilisieren, nicht mehr und nicht weniger. Das geht nicht ohne souveränes Denken der Vielen. 

    Dieses Buch Dieter Kleins (geb. 1931) baut auf einem produktiven, langen intellektuellen Leben auf. Zitate von Marx, Bloch und Rosa Luxemburg, Becher, Hesse und Brecht, Ota Šik und Wolfgang Streeck sind Ausgangspunkt oder Illustration wichtiger eigener Thesen. Zentrale Aussagen werden als Merksätze formuliert, am Rand angestrichen und in angemessener Weise wiederholt. Die Widersprüche und Konflikte der Geschichte linker Politik werden deutlich herausgearbeitet. Im ersten Kapitel werden linke Niederlagen und epochale Irrtümer zur Grundlage kritischen Nachdenkens gemacht, „negative historische Erfahrungen“ linker Politik werden resümiert. Von westeuropäischen Sozialprojekten über die russische Revolution von 1917 hin zur aktuellen lateinamerikanischen Suche nach einer gerechten Gesellschaft: überall ein Scheitern, und das nicht nur an äußeren Feinden. Aber umsonst waren diese Kämpfe nicht. Sie bieten die Möglichkeit zu lernen. Klein unterstreicht seine eigene Herangehensweise: „Hier wird die Ansicht vertreten, dass eine sozial-ökologische Regulationsweise möglich ist“ (17).

    Untermauert wird diese Position durch den Verweis auf drängende globale Probleme, wie die Klimakrise, sowie auf historische Erfahrungen von gelungenem gesteuerten gesellschaftlichen Wandel: auf den New Deal und die US-Kriegswirtschaft unter Präsident Roosevelt sowie die chinesische Erfolgsgeschichte seit den Reformen Deng Xiaopings am Ende der 1970er Jahre. Damit ist ein wichtiges Merkmal der Kleinschen Argumentationsweise benannt, die ausführliche Nutzung historischer und aktueller Erfahrungen als ein Mittel zur Überprüfung der Realisierbarkeit der eigenen Vorschläge. 

    Das zweite Kapitel referiert in sehr gut lesbarer Weise den gegenwärtigen linken Regulationsdiskurs in 15 Abschnitten – von Erik Olin Wright über Elinor Ostrom bis Rainer Land und Hans Thie. Die wichtigste eigene These wird, gestützt vor allem auf Aussagen des marxistischen Soziologen Wright, gleich zu Beginn als Merksatz formuliert: „Im Zentrum eines linken Regulationsdiskurses steht […], welche Bedeutung der gesellschaftlichen Planung, dem Marktmechanismus und dem Handeln zivilgesellschaftlicher demokratischer Akteure in einer zukünftigen progressiven Regulationsweise zukommen sollte.“ (53). Angesichts der Zukunftsblockade durch die kapitalistische Fixierung auf den möglichen maximalen Gegenwartsprofit erhält die öffentliche Planung eine zentrale Bedeutung für die zukünftige solidarische Regulation.

    Im folgenden Kapitel wird das Zusammenspiel dieser drei Elemente ausführlich, auf 80 Seiten, beschrieben. Erst aus dem Wechselspiel jener Kräfte entstehe das Gemeinwohl. Zentral dabei ist ein durch Politik und Gesellschaft regulierter Markt. In die Analyse der möglichen Funktion des Marktmechanismus innerhalb einer solidarischen Regulation bringt der Autor sein umfangreiches Wissen über Plan und Markt in dem gescheiterten Sozialismusversuch ein. Ein Lernprozess ist nötig, der die eigenen Erfahrungen nicht vergisst, sowohl nicht die negativen der administrativen Detailplanung als auch die aus den Experimenten mit Marktmechanismen etwa im jugoslawischen Sozialismus oder in der Tschechoslowakei nach 1965 (116ff.). Dabei gesellt sich bei Klein zu den drei Elementen einer progressiven Regulationsweise, wenn man denn zählen will, noch ein viertes, zu begreifen auch als deren allgemeine Voraussetzung: eine Änderung der Eigentumsverhältnisse in der Wirtschaft. Auch das ist ein Ergebnis eines produktiven Umgangs mit dem eigenen Erbe, welcher aus der Ablehnung einer umfassenden Verstaatlichung im Staatssozialismus keine Ignoranz gegenüber der zentralen Stellung der Eigentumsfrage ableitet. 

    Klein spricht von einem „Bruch in den Eigentumsverhältnissen. [...] Jene Wirtschaftsressourcen, deren Beherrschung über den Grundzustand und die Zukunft der Gesellschaft und der Naturverhältnisse entscheiden, dürfen nicht in den Händen enger Kreise der privatkapitalistischen Machteliten bleiben“ (153, 154). Entstehen sollen auf diese Weise „hybride Eigentumsverhältnisse mit einem starken Gewicht von Gemeineigentum“ (156). Dabei allerdings ginge es nicht um formelles Staatseigentum, sondern um verschiedene Formen von Gemeineigentum. Er stützt sich, wie auch Michael Brie, umfänglich auf Ostroms Verständnis von „Commons“. 

    Das größte Rätsel linker Politik ist, warum sich trotz verbreiteter Zustimmung zur Richtung der nötigen Veränderungen so wenig tut. Warum handeln so verdammt wenige entsprechend ihrer Einsichten? Wie kann man, um Marx zu paraphrasieren, „die Verhältnisse zum Tanzen bringen“? Diesem Rätsel widmet sich Klein im letzten Kapitel, das wie folgt überschrieben ist: „Wie kann der Übergang zu progressiver Transformation und Regulationsweise in Gang kommen?“ Die Antwort wird ausführlich und differenziert gegeben, verkürzt kann sie wie folgt zusammengefasst werden: durch Druck von unten und unter Nutzung von Bewegungsräumen, die in der Auseinandersetzung zwischen verschiedenen Gruppen einer sich ausdifferenzierenden herrschenden Klasse entstehen. Im Abschnitt über eine doppelte Transformation wird erläutert, dass sich die beiden bisherigen Konzepte von Veränderungen im und gegen den Kapitalismus im Interesse der Bevölkerungsmehrheit, Revolution und Reform, jede für sich als problematisch erwiesen haben (224). Stattdessen gehe es darum, in einem innersystemischen progressiven Wandlungsprozess „Einstiegsprojekte in eine systemüberschreitende“ Transformation zu verwirklichen. Und in diesem Prozess würden notwendigerweise Eigentumsfragen aufgeworfen, es müssten also „heilige Kühe“ geschlachtet werden (225). An Protestaktionen zur Wohnungsfrage und im Gesundheitswesen wurde schon früher (Kap. 3.4.) dargestellt, wie so etwas aussehen könnte. 

    Das Buch bringt noch viele Anregungen zum Weiterdenken für den interessierten Leser. Dem Rezensenten selbst brachte es die Möglichkeit, über Fragen, die ihn schon längere Zeit beschäftigen, auf neuer Grundlage nachzudenken, etwa über den Platz einer Demokratisierung der Demokratie in einer sozialen und ökologischen Transformation und Wege zu einer deutlichen Verbreiterung der sozialen Basis linker Politik. Offen bleibt die Frage – weil kein Buch eine Antwort darauf geben kann – ob die für eine ökologische und soziale Transformation nötige radikale Veränderung unserer Wirtschafts- und Lebensweise noch rechtzeitig verwirklicht werden kann, um einen drohenden Kollaps der menschlichen Zivilisation abzuwenden. Dieter Kleins Antwort darauf jedenfalls steckt im Zitat Hermann Hesses über den Moralphilosophen Kung Fu Tse: „ist das nicht der, der genau weiß, dass es nicht geht und es trotzdem tut?“. So hat der Autor jenes Buches es immer gehalten. \\\bigskip

    \fbox{
        \parbox[c][][c]{0.4\textwidth}{
            Dieter Klein: Regulation in einer solidarischen Gesellschaft. Wie eine sozial-ökonomische Transformation gelingen könnte. Hamburg: VSA 2022, 265 Seiten. 
        }
    }
\end{multicols*}