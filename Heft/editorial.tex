\bdieditorial{Johanna Wischner, Thomas Müller}

\begin{multicols*}{2}
\noindent „Auf in die Provinz!“, so lautet der leicht ironisch gemeinte Schlachtruf, mit dem wir den Schwerpunkt dieses Hefts überschreiben. Natürlich stellt sich die Frage: Was könnte ein Hauptstadtjournal, ein Journal, das in der einzigen echten Metropole Deutschlands (so zumindest einige spitze Zungen) zu Hause ist, überhaupt zum Thema Provinz sagen?
Zunächst einmal nicht viel, so erscheint es zumindest, ist doch die Forschungsperspektive in vielen Sozial- und Geisteswissenschaften metropolitan. Oft genug wird der ländliche Raum als „das Andere“ der Metropole kon­struiert. In und nach der Pandemie, in der die Enge der Großstadt ebenso wie die Möglichkeiten mobilen Arbeitens zu Themen des öffentlichen Diskurses wurden zeitweise größer, scheint allerdings das Interesse am Ländlichen gestiegen zu sein. Man hatte gar den Eindruck einer „Landeuphorie“.

Dennoch: Ist der sich angesichts steigender Immobilienpreise immer weiter ausdehnende Speckgürtel Berlins nicht letztlich nichts weiter als die Provinz der Großstadt, und zwar in einem sehr römischen Sinne: abhängig von ihr, auf sie gerichtet und ihr letztlich untergeordnet? Wo ist die Provinz, die plötzlich von einer großen Zahl digitaler Nomaden besiedelt zu sein schien, in ihrem Eigen-Sinn? Was macht sie aus, was ist ihre Qualität, unabhängig von ihrer „zuarbeitenden“ Funktion für die Metropole?

Fragen wie diese haben uns angeregt, das Thema „Provinz“, gerade als Hauptstadtjournal, in den Blick zu nehmen. Allerdings ist dieser Gegenstand einer, der sich entschieden entzieht. Es gibt zwar durchaus sozial- und geisteswissenschaftliche Forschung 

Dennoch: Ist der sich angesichts steigender Immobilienpreise immer weiter ausdehnende Speckgürtel Berlins nicht letztlich nichts weiter als die Provinz der Großstadt, und zwar in einem sehr römischen Sinne: abhängig von ihr, auf sie gerichtet und ihr letztlich untergeordnet? Wo ist die Provinz, die plötzlich von einer großen Zahl digitaler Nomaden besiedelt zu sein schien, in ihrem Eigen-Sinn? Was macht sie aus, was ist ihre Qualität, unabhängig von ihrer „zuarbeitenden“ Funktion für die Metropole?

Fragen wie diese haben uns angeregt, das Thema „Provinz“, gerade als Hauptstadtjournal, in den Blick zu nehmen. Allerdings ist dieser Gegenstand einer, der sich entschieden entzieht. Es gibt zwar durchaus sozial- und geisteswissenschaftliche Forschung \textit{über} die Provinz, weniger aber \textit{aus} der Provinz, sofern man keinen sehr engen Provinzbegriff vertritt, der ganz Deutschland, abgesehen von Berlin, zur Provinz erklärt. Aber genau an dieser Stelle drängen sich schon Fragen auf: Was ist überhautextit{über} die Provinz, weniger aber \textit{aus} der Provinz, sofern man keinen sehr engen Provinzbegriff vertritt, der ganz Deutschland, abgesehen von Berlin, zur Provinz erklärt. Aber genau an dieser Stelle drängen sich schon Fragen auf: Was ist überhaupt die Provinz, und \textit{wo} ist sie? Sind es nur die Dörfer und vielleicht noch die Kleinstädte? Oder zählen die Provinz, und \textit{wo} ist sie? Sind es nur die Dörfer und vielleicht noch die Kleinstädte? Oder zählen auch Mittelstädte zuuch Mittelstädte zur Provinz? Ist Provinz überhaupt eine rein räumliche Kategorie? – Wie Raum und Geisteshaltung einander bedingen, wird in einigen der Artikel dieses Schwerpunkts diskutiert. Die Spannung zwischen Raum und Geist ist zentral, wenn man das Schillern und Changieren des Motivs „Provinz“ verstehen will. Es weist auf eine räumliche Verfasstheit hin, weist im Begriff der Provinzialität aber auch über diese hinaus. Geistesgeschichtlich kann auch von „Provinzen des Denkens“ gesprochen werden, von abseitigen Gegenden, durch die nie die Hauptströmungen irgendeiner Wissenschaftsdisziplin flossen und die dennoch unseren akademischen Reichtum und die Vielfalt des Denkens ausmachen.

Dass wissenschaftliche Forschung überwiegend in Metropolen stattfindet und aufgrund der Verfasstheit der akademischen Strukturen, dort, wo sie in kleineren Städten situiert ist, doch letztlich oftmals von Forschenden ausgeführt wird, die aus Großstädten pendeln oder nur vorübergehend in „der Provinz“ wohnen, wirft ein politisches Schlaglicht auf die Frage, wo die Provinz in der Forschung eigentlich bleibt. Die Frage nach der Provinz als Forschungsgegenstand sowie als (oftmals fehlende) Perspektive und Standpunkt des Forschens verweist eindrücklich auf die Situiertheit unseres Denkens und unserer Forschung – nicht nur in geistigen Kategorien, auch räumlich sind sie immer verortet. Diese Situiertheit zu reflektieren, gehört dank Wissenschaftssoziologie und -geschichte zum sozial- und geisteswissenschaftlichen \textit{Common Sense}. Allerdings bleibt sie oftmals unreflektiert, wenn es um die metropolitane gegenüber der provinziellen Perspektive geht.

Zwar ist die Vorstellung wissenschaftlicher „Objektivität“ in einem strengen Sinne im Zuge der Wissenschaftskritik des 20. Jahrhunderts obsolet geworden. Aber die Loslösung von ihrem Entstehungskontext ist nach wie vor Bestreben und Ziel von Forschung und ihre „Ortlosigkeit“ daher nur die logische Folge. Allerdings besteht so immer die Gefahr, dass die Perspektive der Metropole als Perspektive unsichtbar und damit unausgesprochen zur Norm wird. Ähnliches ließe sich vermutlich auch über andere Bereiche von Kultur und Öffentlichkeit sagen, wie den überregionalen Journalismus, der den öffentlichen Diskurs in anderer Weise prägt als wissenschaftliche Forschung. Er wird ebenfalls in Großstädten gemacht und thematisiert den ländlichen Raum nicht selten auf fast schon ethnologische Weise als das faszinierende und/oder bedrohliche Andere. 

Die Beiträge des Schwerpunkts befassen sich auf verschiedene Weise mit Räumen der Provinz, des Provinziellen. Immer aber klingt durch: Die Provinz ist nie nur räumlich gedacht, sie ist immer auch ein soziales, ein politisches, ein historisches, ein geistiges und auch ein ökonomisches Phänomen. Ohne die Spannung zwischen Metropole und Provinz (wie auch immer diese dann genau bestimmt sein mögen) kommt allerdings kein Beitrag aus. Diese beiden Pole bedingen einander und erst aus ihrer Beziehung ergeben sich \looseness=1 Selbstverständnisse, -beschreibungen und -verortungen und damit letztlich auch gesellschaftliche Verhältnisse. Die Eigensinnigkeit der Provinz ist also vermutlich nicht zu erfassen ohne eine Abwertung des Metropolitanen – eine analytisch nicht sehr befriedigende Perspektive. Insofern liegt es nahe, herauszuarbeiten, wie sich diese beiden Pole aufeinander beziehen, wie ihr Spannungsverhältnis Entwicklungen bedingt und auch Bewertungs- und Analyseraster hervorbringt.

Zum Auftakt untersucht Tobias Becker das historische Missverhältnis zwischen Metropole und Provinz. Er betont die Relationalität der beiden Begriffe, die aber wegen der hegemonialen metropolitanen Perspektiven letztlich zu Ungunsten des Ländlichen und der Provinz ausfallen. Trotz der zunehmenden Vernetzung der Welt und des Zurücktretens des physischen Orts in Zeiten globaler Kommunikation sei der Topos des Provinziellen nicht verschwunden. Allerdings wurde, so Becker, das Metropole-Provinz-Verhältnis vor dem Hintergrund der Globalisierung neu justiert.

Die nächsten beiden Texte behandeln grundlegende philosophische Fragen anhand der Dichotomie Metropole/Provinz. Die Verortetheit unseres Denkens kann, so scheint es, auf anthropologischer und metaphysischer Ebene nicht durch Vernetzung und Digitalisierung zum Verschwinden gebracht werden. Zanan Akin stellt die Frage, worin Metropolität und Provinzialität überhaupt noch bestehen können, wenn der Lebensort durch ortsungebundenes Arbeiten immer weniger bedeutsam und der Lebensstil entscheidend wird. Anhand einer auf Heidegger rekurrierenden kritischen Auseinandersetzung mit Andreas Reckwitz’ Konzeption der „Gesellschaft der Singularitäten“ wird die „Erlebbarkeit“ als Schlüsselbegriff herausgearbeitet, der eine neue Annäherung an das Verhältnis zwischen Provinz und Metropole ermöglichen könnte. Ebenfalls mit Heidegger, dem Philosophen der Provinz par excellence – der reaktionäre, nationalistische Unterton seiner Provinzbegeisterung muss dabei immer mitgedacht werden –, arbeitet Michael Meyer-Albert, wobei er diesem Hannah Arendt und ihre Emphase der Urbanität gegenüberstellt. Provinzialität und Urbanität werden als zueinander in einem Spannungsverhältnis stehende Modi der existenziellen Weltoffenheit dargestellt, die auch und gerade in einer globalisierten Welt notwendig bleiben.

Am konkreten Fall und an empirischem Material orientieren sich die drei folgenden Beiträge: Zunächst erinnert Bernd Belina an die westdeutsche Provinz-Debatte der 1970er Jahre. 

Er präsentiert eine strukturierte Collage aus „Fundstücken“, die auch für gegenwärtige Debatten aufschlussreich sein können: Die Provinz als Raumkategorie in ihrem heutigen Sinn ist Produkt der räumlich ungleichen Entwicklung des Kapitalismus. Es sind diese räumlichen Verhältnisse, die Provinzialismus – verstanden als apodiktisches Setzen des Eigenen gegen das Fremde – begünstigen, aber nicht bedingen. Daher lässt sich Provinzialismus häufiger, aber nicht nur, in der Provinz finden. Den Nexus zwischen Fußball und Provinz arbeitet Philipp Didion am Beispiel von Fußballstadien heraus. Er zeigt, dass neben dem oftmals mit dem Fußball assoziierten aggressiven Lokalpatriotismus weitere Bedeutungsschichten existieren. In einer vergleichenden Analyse zwischen Frankreich und der Bundesrepublik Deutschland deutet sich an, dass Fußballstadien beides sein können: Orte kreativer Re-Provinzialisierung mittels provinz- und heimatpraktischer Aneignungsprozesse und Orte der De-Provinzialisierung mittels sukzessiver Horizont- und Funktionserweiterungen. Am Beispiel von Gütersloh zeichnen Joana Gelhart, Christoph Lorke und Tim Zumloh in ihrer Fallstudie die Entwicklung von der Mittelstadt zur Großstadt nach. Sie gehen jenen Neupositionierungen nach, die sich in den Handlungsfeldern städtischer Repräsentations-, Infrastruktur- und Migrationspolitik zwischen Provinz und Großstadt bewegten. Ihrer Untersuchung liegt ein performatives Verständnis zugrunde, wonach „Provinz“ und „Großstadt“ keine faktischen Größen darstellen, sondern stets diskursiv und habituell hervorgebracht und individuell angeeignet werden.

Provinz? Ist Provinz überhaupt eine rein räumliche Kategorie? – Wie Raum und Geisteshaltung einander bedingen, wird in einigen der Artikel dieses Schwerpunkts diskutiert. Die Spannung zwischen Raum und Geist ist zentral, wenn man das Schillern und Changieren des Motivs „Provinz“ verstehen will. Es weist auf eine räumliche Verfasstheit hin, weist im Begriff der Provinzialität aber auch über diese hinaus. Geistesgeschichtlich kann auch von „Provinzen des Denkens“ gesprochen werden, von abseitigen Gegenden, durch die nie die Hauptströmungen irgendeiner Wissenschaftsdisziplin flossen und die dennoch unseren akademischen Reichtum und die Vielfalt des Denkens ausmachen.

Dass wissenschaftliche Forschung überwiegend in Metropolen stattfindet und aufgrund der Verfasstheit der akademischen Strukturen, dort, wo sie in kleineren Städten situiert ist, doch letztlich oftmals von Forschenden ausgeführt wird, die aus Großstädten pendeln oder nur vorübergehend in „der Provinz“ wohnen, wirft ein politisches Schlaglicht auf die Frage, wo die Provinz in der Forschung eigentlich bleibt. Die Frage nach der Provinz als Forschungsgegenstand sowie als (oftmals fehlende) Perspektive und Standpunkt des Forschens verweist eindrücklich auf die Situiertheit unseres Denkens und unserer Forschung – nicht nur in geistigen Kategorien, auch räumlich sind sie immer verortet. Diese Situiertheit zu reflektieren, gehört dank Wissenschaftssoziologie und -geschichte zum sozial- und geisteswissenschaftlichen \textit{Common Sense}. Allerdings bleibt sie oftmals unreflektiert, wenn es um die metropolitane gegenüber der provinziellen Perspektive geht.

Zwar ist die Vorstellung wissenschaftlicher „Objektivität“ in einem strengen Sinne im Zuge der Wissenschaftskritik des 20. Jahrhunderts obsolet geworden. Aber die Loslösung von ihrem Entstehungskontext ist nach wie vor Bestreben und Ziel von Forschung und ihre „Ortlosigkeit“ daher nur die logische Folge. Allerdings besteht so immer die Gefahr, dass die Perspektive der Metropole als Perspektive unsichtbar und damit unausgesprochen zur Norm wird. Ähnliches ließe sich vermutlich auch über andere Bereiche von Kultur und Öffentlichkeit sagen, wie den überregionalen Journalismus, der den öffentlichen Diskurs in anderer Weise prägt als wissenschaftliche Forschung. Er wird ebenfalls in Großstädten gemacht und thematisiert den ländlichen Raum nicht selten auf fast schon ethnologische Weise als das faszinierende und/oder bedrohliche Andere. 

Die Beiträge des Schwerpunkts befassen sich auf verschiedene Weise mit Räumen der Provinz, des Provinziellen. Immer aber klingt durch: Die Provinz ist nie nur räumlich gedacht, sie ist immer auch ein soziales, ein politisches, ein historisches, ein geistiges und auch ein ökonomisches Phänomen. Ohne die Spannung zwischen Metropole und Provinz (wie auch immer diese dann genau bestimmt sein mögen) kommt allerdings kein Beitrag aus. Diese beiden Pole bedingen einander und erst aus ihrer Beziehung ergeben sich Selbstverständnisse, -beschreibungen und -verortungen und damit letztlich auch gesellschaftliche Verhältnisse. Die Eigensinnigkeit der Provinz ist also vermutlich nicht zu erfassen ohne eine Abwertung des Metropolitanen – eine analytisch nicht sehr befriedigende Perspektive. Insofern liegt es nahe, herauszuarbeiten, wie sich diese beiden Pole aufeinander beziehen, wie ihr Spannungsverhältnis Entwicklungen bedingt und auch Bewertungs- und Analyseraster hervorbringt.

Zum Auftakt untersucht Tobias Becker das historische Missverhältnis zwischen Metropole und Provinz. Er betont die Relationalität der beiden Begriffe, die aber wegen der hegemonialen metropolitanen Perspektiven letztlich zu Ungunsten des Ländlichen und der Provinz ausfallen. Trotz der zunehmenden Vernetzung der Welt und des Zurücktretens des physischen Orts in Zeiten globaler Kommunikation sei der Topos des Provinziellen nicht verschwunden. Allerdings wurde, so Becker, das Metropole-Provinz-Verhältnis vor dem Hintergrund der Globalisierung neu justiert.

Die nächsten beiden Texte behandeln grundlegende philosophische Fragen anhand der Dichotomie Metropole/Provinz. Die Verortetheit unseres Denkens kann, so scheint es, auf anthropologischer und metaphysischer Ebene nicht durch Vernetzung und Digitalisierung zum Verschwinden gebracht werden. Zanan Akin stellt die Frage, worin Metropolität und Provinzialität überhaupt noch bestehen können, wenn der Lebensort durch ortsungebundenes Arbeiten immer weniger bedeutsam und der Lebensstil entscheidend wird. 

Anhand einer auf Heidegger rekurrierenden kritischen Auseinandersetzung mit Andreas Reckwitz’ Konzeption der „Gesellschaft der Singularitäten“ wird die „Erlebbarkeit“ als Schlüsselbegriff herausgearbeitet, der eine neue Annäherung an das Verhältnis zwischen Provinz und Metropole ermöglichen könnte. Ebenfalls mit Heidegger, dem Philosophen der Provinz par excellence – der reaktionäre, nationalistische Unterton seiner Provinzbegeisterung muss dabei immer mitgedacht werden –, arbeitet Michael Meyer-Albert, wobei er diesem Hannah Arendt und ihre Emphase der Urbanität gegenüberstellt. Provinzialität und Urbanität werden als zueinander in einem Spannungsverhältnis stehende Modi der existenziellen Weltoffenheit dargestellt, die auch und gerade in einer globalisierten Welt notwendig bleiben.

Am konkreten Fall und an empirischem Material orientieren sich die drei folgenden Beiträge: Zunächst erinnert Bernd Belina an die westdeutsche Provinz-Debatte der 1970er Jahre. Er präsentiert eine strukturierte Collage aus „Fundstücken“, die auch für gegenwärtige Debatten aufschlussreich sein können: Die Provinz als Raumkategorie in ihrem heutigen Sinn ist Produkt der räumlich ungleichen Entwicklung des Kapitalismus. Es sind diese räumlichen Verhältnisse, die Provinzialismus – verstanden als apodiktisches Setzen des Eigenen gegen das Fremde – begünstigen, aber nicht bedingen. Daher lässt sich Provinzialismus häufiger, aber nicht nur, in der Provinz finden. Den Nexus zwischen Fußball und Provinz arbeitet Philipp Didion am Beispiel von Fußballstadien heraus. Er zeigt, dass neben dem oftmals mit dem Fußball assoziierten aggressiven Lokalpatriotismus weitere Bedeutungsschichten existieren. In einer vergleichenden Analyse zwischen Frankreich und der Bundesrepublik Deutschland deutet sich an, dass Fußballstadien beides sein können: Orte kreativer Re-Provinzialisierung mittels provinz- und heimatpraktischer Aneignungsprozesse und Orte der De-Provinzialisierung mittels sukzessiver Horizont- und Funktionserweiterungen. Am Beispiel von Gütersloh zeichnen Joana Gelhart, Christoph Lorke und Tim Zumloh in ihrer Fallstudie die Entwicklung von der Mittelstadt zur Großstadt nach. Sie gehen jenen Neupositionierungen nach, die sich in den Handlungsfeldern städtischer Repräsentations-, Infrastruktur- und Migrationspolitik zwischen Provinz und Großstadt bewegten. Ihrer Untersuchung liegt ein performatives Verständnis zugrunde, wonach „Provinz“ und „Großstadt“ keine faktischen Größen darstellen, sondern stets diskursiv und habituell hervorgebracht und individuell angeeignet werden.

Der Schwerpunkt wird abgerundet durch zwei Projektvorstellungen, die in unterschiedlichen Anteilen eine Verortung in „der Provinz“ betonen und zugleich die Potenziale des spezifisch ländlichen Blicks und der ländlichen Lebensweise heben wollen. Ricardo Kaufer versteht die Provinz als „Projektraum“ für eine Nachhaltigkeitstransformation von unten. Er stellt verschiedene Projekte vor, die zur ökonomischen und ökologischen gesellschaftlichen Transformation beitragen wollen, und argumentiert, dass der ländliche Raum ein günstiger Nährboden für diese Bestrebungen ist. Zugleich fordert Kaufer, dass moderne Staatlichkeit die Zugänglichkeit zur und Gestaltbarkeit der Provinz stärker als öffentliches Gut organisieren sollte. Kenneth Anders rückt den Begriff des Eigensinns ins Zentrum seiner Überlegungen: Er weist darauf hin, dass Provinzen nicht aus sich heraus bestehen, sondern in Beziehung zu einem Staat oder Ballungsraum zu sehen sind, ohne deswegen in den Interessen des Zentrums aufzugehen. Provinzieller Eigensinn, so Anders, entfaltet sich durch regionale Kommunikation – im Interesse des ländlichen Raums, aber auch als Korrektiv in der Gesellschaft. Diese Dialektik arbeitet der Autor am Beispiel des Oderbruchs und eines dort entwickelten Ansatzes regionaler Selbstbeschreibung heraus.

Außerhalb des Schwerpunkts analysiert Roger Woods „Das Echolot. Abgesang ’45“, den letzten Band von Walter Kempowskis „kollektivem Tagebuch“ des Zweiten Weltkrieges. Beim Vergleich von Kempowskis Originalunterlagen mit den Auszügen, die im „Abgesang ’45“ veröffentlicht wurden, zeigt sich, dass Kempowski das Material, das er erhalten hatte, so arrangierte, überarbeitete und rekontextualisierte, dass die heterogenen Erinnerungen verschiedener Individuen gegenüber der Darstellung kollektiven deutschen Leids in den Hintergrund treten. Das nicht zuletzt in der Corona-Pandemie in Bewegung geratene Verhältnis von Arbeit und Freizeit ist der Ausgangspunkt für Gregor Ritschel, nach der Geschichte und Aktualität des Begriffs der freien Zeit zu fragen. Ritschel zufolge ist freie Zeit eine politische Idee, die es verdient, stärker beachtet und in ihrem Eigensinn gewürdigt zu werden. Sein Beitrag endet mit einem Plädoyer für eine neue Kultur der freien Zeit, die mehr ist als Erholung von der Erwerbsarbeit.
Der Schwerpunkt wird abgerundet durch zwei Projektvorstellungen, die in unterschiedlichen Anteilen eine Verortung in „der Provinz“ betonen und zugleich die Potenziale des spezifisch ländlichen Blicks und der ländlichen Lebensweise heben wollen. Ricardo Kaufer versteht die Provinz als „Projektraum“ für eine Nachhaltigkeitstransformation von unten. Er stellt verschiedene Projekte vor, die zur ökonomischen und ökologischen gesellschaftlichen Transformation beitragen wollen, und argumentiert, dass der ländliche Raum ein günstiger Nährboden für diese Bestrebungen ist. Zugleich fordert Kaufer, dass moderne Staatlichkeit die Zugänglichkeit zur und Gestaltbarkeit der Provinz stärker als öffentliches Gut organisieren sollte. Kenneth Anders rückt den Begriff des Eigensinns ins Zentrum seiner Überlegungen: Er weist darauf hin, dass Provinzen nicht aus sich heraus bestehen, sondern in Beziehung zu einem Staat oder Ballungsraum zu sehen sind, ohne deswegen in den Interessen des Zentrums aufzugehen. Provinzieller Eigensinn, so Anders, entfaltet sich durch regionale Kommunikation – im Interesse des ländlichen Raums, aber auch als Korrektiv in der Gesellschaft. Diese Dialektik arbeitet der Autor am Beispiel des Oderbruchs und eines dort entwickelten Ansatzes regionaler Selbstbeschreibung heraus.

Außerhalb des Schwerpunkts analysiert Roger Woods „Das Echolot. Abgesang ’45“, den letzten Band von Walter Kempowskis „kollektivem Tagebuch“ des Zweiten Weltkrieges. Beim Vergleich von Kempowskis Originalunterlagen mit den Auszügen, die im „Abgesang ’45“ veröffentlicht wurden, zeigt sich, dass Kempowski das Material, das er erhalten hatte, so arrangierte, überarbeitete und rekontextualisierte, dass die heterogenen Erinnerungen verschiedener Individuen gegenüber der Darstellung kollektiven deutschen Leids in den Hintergrund treten. Das nicht zuletzt in der Corona-Pandemie in Bewegung geratene Verhältnis von Arbeit und Freizeit ist der Ausgangspunkt für Gregor Ritschel, nach der Geschichte und Aktualität des Begriffs der freien Zeit zu fragen. Ritschel zufolge ist freie Zeit eine politische Idee, die es verdient, stärker beachtet und in ihrem Eigensinn gewürdigt zu werden. Sein Beitrag endet mit einem Plädoyer für eine neue Kultur der freien Zeit, die mehr ist als Erholung von der Erwerbsarbeit.

\end{multicols*}

\pagebreak
