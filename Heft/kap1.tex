\bdichapter{Roger Woods}{%
Walter Kempowskis \enquote{Das Echolot. Abgesang '45}%
}{%
Walter Kempowskis \enquote{Das Echolot.\\Abgesang '45}%
}{%
Vom Archiv zur Druckfassung\endnote{Übersetzte und überarbeitete Fassung von „Walter Kempowski’s Das Echolot. Abgesang ’45. From Archive to Print“, erschienen in: German Life and Letters 66 (2013), Heft 4, S. 416-431 (German Life and Letters © John Wiley \& Sons Ltd.). Mit freundlicher Genehmigung des Verlags.}%
}
    
\begin{multicols*}{2}

\noindent „Das Echolot. Abgesang ’45“, der letzte Band von Walter Kempowskis zehnbändigem „kollektivem Tagebuch“ des Zweiten Weltkriegs für die Zeit von Juni 1941 bis Mai 1945, erschien 2005 und wurde in der deutschen Presse vielfach mit großer Begeisterung aufgenommen.\endnote{„Der Spiegel“ ernannte Kempowski zu einem „Meister der Form und der Proportion“ und zu einem „würdigen Kandidaten“ für den Georg-Büchner-Preis und den Friedenspreis des Deutschen Buchhandels (Hage 2005: 168). Die „Süddeutsche Zeitung“ wertete „Abgesang ’45“ als „Triumph“, der eines der „ambitioniertesten“ und „beeindruckendsten Unternehmen der deutschen Literaturgeschichte“ vollende (Seibt 2005). Frank Schirrmacher (1993) bezeichnete es in der „Frankfurter Allgemeine Zeitung“ als „eine der größten Leistungen der Literatur unseres Jahrhunderts“. Und „Die Welt“ erklärte Kempowski zu einem „Herakles“ „literarischer Bricolage und historiographischer Collage“ (Werner 2005).} Wie die vorangegangenen neun Bände, die zu Bestsellern wurden und Kempowski zahlreiche Literaturpreise und Ehrungen einbrachten, besteht es aus Auszügen aus Tagebüchern, Briefen, veröffentlichten und unveröffentlichten Memoiren, Zeitungen, Texten von Radiosendungen und Reden, Todesanzeigen und militärischen Mitteilungen. Diese Auszüge stammen aus den Schriften bekannter Politiker, Intellektueller und Journalisten sowie von Soldaten und KZ-Häftlingen, Kriegsgefangenen und anderen, die in keiner Weise berühmt waren, aber über ihre Erfahrungen in den Kriegsjahren berichtet haben. 

Ein \textbf{großer Teil} des \textit{unveröffentlichten Materials}, das in den einzelnen Bänden von „Das Echolot“ zitiert wird, war Kempowski auf seine Anzeigen in der deutschen Presse seit Anfang der 1970er Jahre zugesandt worden, in denen er die Leser um Fotos aus der Zeit bis 1950, unveröffentlichte Tagebücher, Briefe und Autobiographien bat. Diese Sammlung von rund 8.000 biographischen und weiteren Dokumenten bildet das sogenannte Biografienarchiv von Kempowski, es umfasst rund 400 Regalmeter und befindet sich heute im Archiv der Akademie der Künste in Berlin.\endnote{Die Bestände des Kempowski-Archivs sind im Katalog des Archivs der Akademie der Künste detailliert beschrieben: https://archiv.adk.de/bigobjekt/37026.} Aus der Sammlung präsentiert Kempowski in „Abgesang ’45“ Auszüge für den Zeitraum vom 20. April bis 9. Mai 1945; der Band bildet den Abschluss von Kempowskis umfangreichem, zwanzigjährigem „Echolot“-Projekt (vgl. Kempowski 2005a: 5).

In diesem Artikel wird vor dem Hintergrund der Theoriediskussionen zur Autobiographie und des wachsenden Interesses der Historiker an der subjektiven Erfahrung von Geschichte untersucht, wie Kempowski sein Material organisierte. Dabei werden die Materialien, die Kempowski von den jeweiligen Autoren oder ihren Angehörigen für sein Biografienarchiv erhielt, mit den in „Abgesang ’45“ veröffentlichten Texten verglichen.

Helmut Schmitz (2007a: 5) konstatiert für die jüngste Zeit eine Verschiebung der historiographischen und populären Diskurse von einer Geschichte der „harten Fakten“ hin zur „Erzählung“. Ähnlich charakterisiert Norbert Frei die Zeit von 1995 bis 2005 als das Jahrzehnt der „Zeitzeugen“ und der „gefühlten Geschichte“ (zit. n. Schmitz 2007a: 5f.). Gabriele Jancke benennt in ihrer Studie „Autobiographie als soziale Praxis“ die allgemeineren Gründe für das wachsende Interesse in der Wissenschaft an der individuellen Sichtweise, indem sie beschreibt, wie die Lektüre autobiographischer Texte ihre „Sicht von Geschichte verändert“ habe. „Nach der Lektüre von mehreren Hundert solcher Schriften“ sei ihre Schlussfolgerung, die „Vielfalt von Perspektiven“ der autobiographischen Texte untergrabe die Plausibilität von Verallgemeinerungen, denn in ihr werde sichtbar, wie verschiedene Gruppen – Minderheiten und Mehrheiten, Intellektuelle und Menschen mit geringer Bildung, Frauen und Männer – die Gesellschaft auf unterschiedliche Weise erlebten. Jancke zeigt autobiographische Texte namentlich genannter Personen als eine Herausforderung für den Prozess der Verallgemeinerung, der im Mittelpunkt vieler akademischer Arbeiten stehe. Sie schlägt vor, diese Texte daraufhin zu lesen, dass sie Grundlegendes über eine bestimmte Gesellschaft vermitteln (Jancke 2002: VII).\endnote{Saul Friedländers (1997, 2007) zweibändige Geschichte des nationalsozialistischen Deutschlands und der Juden ist ein bekanntes Beispiel für diesen Ansatz. Er stellt die von ihm zitierten jüdischen Tagebuchautoren nicht als passive, verallgemeinerte Objekte der nationalsozialistischen Politik dar, sondern als proaktive individuelle Subjekte. Zur neueren Diskussion über die subjektive Erfahrung von Geschichte vgl. Fulbrook, Rublack 2010.}

In welcher Hinsicht sind diese hohen Erwartungen an die subjektive Erfahrung für Kempowskis Projekt relevant? Sie sind sicherlich ein bedeutender Teil von dessen Rezeption: Dirk Hempel, Kempowskis Biograph, schreibt, aus dem Archiv der unveröffentlichten Autobiographien entstehe in „Das Echolot“ ein „Bild der historischen Ereignisse“ aus der subjektiven Perspektive der Individuen, die sie erlebt haben (Hempel 2006: 44).\endnote{In ähnlicher Weise sieht Jürgen Ritte (2009: 64) Kempowskis Ziel in „Das Echolot“ darin, „Partikel[..] erlebter Realität“ zu sammeln.} Kempowski selbst betont, dass er den ungehörten Stimmen Raum geben wollte. In der Einleitung zum ersten Band von „Das Echolot“ beschreibt er, wie er in Göttingen „einen Haufen Fotos und Briefe auf der Straße liegen“ sah, über den Passanten gingen. Sie entpuppten sich als „die letzte Hinterlassenschaft eines gefallenen Soldaten, Fotos aus Russland und Briefe an seine Braut“. Kempowski sammelte sie ein. Aus dieser konkreten Erfahrung zieht er die Schlussfolgerung, dass wir „den Alten nicht den Mund zuhalten“ sollten, „wenn sie uns etwas erzählen wollen“, und dass wir „ihre Tagebücher nicht in den Sperrmüll geben“ dürfen. Sie sind „an uns gerichtet“, und wir können es uns nicht leisten, die „Erfahrungen ganzer Generationen zu vernichten“: „Es ist unsere Geschichte, die da verhandelt wird.“ (Kempowski 1993: 7).

Obwohl Kempowski von der Geschichte des Einzelnen ausgeht, geht es ihm hier mehr um das Nicht-Vergessen einer kollektiven Vergangenheit als um die Stimme des Individuums. Diese Weise, seinen Standpunkt zu entwickeln, wirft eine weitere Frage für die folgende Analyse auf. Die von Jancke konstatierte Spannung zwischen Verallgemeinerung und subjektiver Erfahrung historischer Ereignisse lässt sich mit Aleida Assmanns Gegenüberstellung von homogenisierten Formen des institutionellen Gedächtnisses, die ein gemeinsames kulturelles Narrativ produzieren, und den heterogenen Erinnerungen von Individuen beschreiben (Assmann 2006: 157). Mit ihr wäre zu fragen: Inwieweit vermittelt „Abgesang ’45“ eine homogenisierte Version der letzten Tage des Zweiten Weltkriegs und inwieweit fängt es heterogene individuelle Erfahrungen ein? Diese Frage stellt sich insbesondere auch in Bezug auf Kempowskis Untertitel für das „Echolot“-Projekt als Ganzes: „ein kollektives Tagebuch“.

Seit Anfang der 2000er Jahre lebt die wissenschaftliche Diskussion über das deutsche Leid und die Deutschen als Opfer des Zweiten Weltkriegs wieder auf.\endnote{Siehe z. B. die Sonderausgabe von „Central European History“, 38. Jg. (2005), Heft 1 „Germans as Victims during the Second World War“, Niven 2006, Schmitz 2007b.} Im Blick auf die große Popularität von „Das Echolot“ und insbesondere von „Abgesang ’45“ – es verkaufte sich von allen Bänden am besten – werde ich auch betrachten, wie Kempowski diese Themen behandelt.

Die Rezeption von Kempowskis „kollektivem Tagebuch“ ist extrem gespalten. Auf der einen Seite stehen jene, die das „Echolot“-Projekt als eine zufällige Sammlung von Texten betrachten. Marcel Reich-Ranicki (2004) etwa urteilte, die ersten vier Bände, die 1993 erschienen, seien nicht mehr als ein chaotisches Durcheinander. Auf der anderen Seite wird Kempowskis gestaltende Hand betont. Carla Damiano (2005: 12) zum Beispiel untersucht „Kempowskis Prinzip der Nebeneinanderstellung zur kontrapunktischen Gliederung der Textauszüge“. Doris Plöschberger (2006: 41) kontrastiert die Einträge in Kempowskis veröffentlichten Tagebüchern, die einander lebhaft widersprechen, mit den Stimmen, die er in „Echolot“ zu einem „Chor“ arrangiere.

\section{Kempowskis Methoden}

\noindent Kempowski selbst verwendet häufig das Bild von sich als Chorleiter, um die Methode zu veranschaulichen, die „Echolot“ zugrunde liegt.\endnote{In einem Spiegel-Interview vom März 2000 erklärt Kempowski, er habe versucht, die kollektive Erfahrung der Ereignisse in „Das Echolot“ zu zeigen, und er beschreibt seine Arbeit wie folgt: „Das alles wird zu einem großen Chor komponiert, der der Wirklichkeit, der damals erlebten Wirklichkeit nahe kommt“ (Hage 2000: 266). Gegen Ende seiner Zeit als Häftling in Bautzen (1948–1956) war Kempowski auch buchstäblich Chorleiter und leitete den Kirchenchor der Gefangenen (vgl. Hempel 2004: 83).} Hempel, der eng mit Kempowski zusammenarbeitete, beschreibt, wie der Autor von einem Aneinanderreihen der Texte zu einer eher interventionistischen Methode der Anordnung für die ersten Bände überging: „Bisher hatte er die Texte nur aneinandergereiht, was bei erneuter Lektüre wenig befriedigte. Deshalb entschloß er sich zu einer Collagierung nach Art seiner Hörspiele, ordnete das Material dialogisch oder verstärkte Eindrücke durch Häufung, ließ Themen abwechseln, wiederkehren, variierte sie.“ (Hempel 2004: 200)\endnote{Angesichts von Kempowskis Ordnung des Materials für „Das Echolot“ geht Jörg Drews’ Gegensatz zwischen der Subjektivität von Kempowskis persönlichen Tagebüchern und der „Selbstverleugnung, welche ihm ‚Das Echolot‘ abverlangte“, eher an der Sache vorbei (Drews 2006: 48).}

In der Einleitung zu „Das Echolot“ beschreibt Kempowski, wie er die Texte zu einem Dialog geformt habe (Kempowski 1993: 7).\endnote{Eine weitere Dimension von Kempowski als Chorleiter ist in seiner Rolle als Leiter eines Teams von „Rechercheuren, Schreibern und Übersetzern“ zu sehen (vgl. Hempel 2007: 112). Kempowski (1999: 905) dankt seinem Team von mehr als dreißig namentlich genannten Personen, aber seine „editorische Notiz“ weist darauf hin, dass er allein die Texte auswählte, die in „Echolot“ eingingen (ebd.: 837).} Die besondere Wirkung des Dialogs, die Kempowski selbst und andere als ein Grundprinzip von „Echolot“ bezeichnen, wird in „Abgesang ’45“ deutlich, etwa wenn es um die Frage geht, was die Deutschen über die Konzentrationslager wussten. Unter dem Datum des 20. April 1945 (Hitlers letzter Geburtstag) finden sich Berichte über das Entsetzen der Deutschen, als sie erfuhren, was in den Lagern geschehen war. Kempowski zitiert Mariela Kuhn, die einen deutschen Kriegsgefangenen in einem Krankenhaus in Oxford besuchte und sich mit ihm über die „Entdeckungen (die schrecklichen) in den Konzentrationslagern Buchenwald und Nordhausen“ unterhielt. Sie kommentiert, dass die Kriegsgefangenen die Berichte nicht glauben könnten, selbst wenn sie die Bilder in den englischen Zeitungen sahen (Kempowski 2005a: 67). 

Ein anderer Deutscher, Ulli S. aus Hamburg, kann die amerikanischen Berichte über die Konzentrationslager kaum glauben, weil sie so „grauenhaft“ sind: „Man möchte sagen: ‚Das stimmt nicht! Das ist Lüge!‘“ (ebd.: 71).\endnote{Siehe auch den Bericht des Schriftstellers Emil Barth unter dem 30. April 1945 über eine Ausgabe des „Kölnischen Kuriers“, eine von der US-amerikanischen Armee auf Deutsch herausgegebene Zeitung, die u. a. „Augenzeugenberichte und photographische Aufnahmen aus den Konzentrationslagern“ enthielt (Kempowski 2005a: 281).} Unter dem Datum des 25. April 1945 zitiert Kempowski jedoch auch den bitteren Bericht von Alfred Kantorowicz, der aus New York darüber schreibt, wie entsetzt die westlichen Demokratien waren, als sie im April 1945 die Konzentrationslager Nazideutschlands „entdeckten“. Kantorowicz kommentiert: „Sie hätten das alles bereits seit zwölf Jahren zur Kenntnis nehmen können, aus Tausenden von Berichten entkommener Opfer, aus dokumentarisch belegten Büchern – dem Braun-Buch zum Beispiel“ (ebd.: 163), dem Bericht über die Brutalität der Nazis, an dem Kantorowicz 1933 selbst mitgearbeitet hatte. Teil des Dialogs, den Kempowski gestaltet, ist hier auch die physische Präsenz von KZ-Häftlingen in deutschen Städten während der letzten Kriegsphase: Ein Arzt schreibt über weibliche Häftlinge aus einem ungenannten Lager, die durch Wittstock marschieren (ebd.: 151), Häftlinge aus dem KZ Sachsenhausen beschreiben ihren „Marsch“ in Richtung Mecklenburg und dass Nachzügler erschossen werden (ebd.: 152f., 164), und der Schriftsteller Wilhelm Hausenstein berichtet, wie ungarische Juden aus dem Lager Dachau zu seinem Haus in Tutzing kamen und um Essen bettelten (ebd.: 249).\endnote{Zu den marschierenden KZ-Häftlingen siehe auch die Einträge ebd.: 274-277.}

Kempowski verdeutlicht mit dem innertextuellen Dialog über die Konzentrationslager, dass die Auszüge aus dem kollektiven Tagebuch nicht isoliert gelesen werden sollten, da sie sich gegenseitig in Frage stellen.\endnote{Bereits 1979 hatte Kempowski die Antworten von dreihundert Deutschen auf seine Frage, ob sie von den NS-Verbrechen wussten, veröffentlicht und „war überrascht von der Freimütigkeit, mit der die Frage bejaht wurde“ (Kempowski 1979: 5).} Kempowskis „Methode, verschiedene Stimmen nebeneinander- und gegenüberzustellen“, brachte den Historiker Peter Fritzsche (2002: 77f.) zu der allgemeinen Schlussfolgerung über die früheren „Echolot“-Bände, dass diese „das Potential eines stärker selbstreflektierenden und kritischen Nutzens zeitgenössischer Geschichte“ zeigten.

Entgegen dieser kritischen Bestimmung weicht der Dialog zwischen den Deutschen in „Abgesang ’45“ jedoch häufiger der Feindseligkeit zwischen Nazis und Nicht-Nazis oder, noch fragwürdiger, zwischen dem deutschen Volk insgesamt und der kleinen Gruppe von Nazi-Führern und -Anhängern, die so dargestellt werden, als hätten sie jenes getäuscht. So lesen wir etwa von einem verwundeten deutschen Soldaten, der von den Amerikanern gefangen genommen und in Schloss Waldeck untergebracht worden war, wie er unfreiwillig zum Teilnehmer einer Feier zu Hitlers Geburtstag wird: „Die Tür geht auf, und drei deutsche Führungsoffiziere kommen rein, grüßen und halten eine Geburtstagsfeier für den Führer. Mit deutschem Gruß! […] Und ich lag im Bett, mit Steckschüssen im Bein und hab’ nicht opponiert, sondern hab’ den Arm gehoben und dachte, ich bin verrückt“ (Kempowski 2005a: 13). Ein dänischer Journalist berichtet, wie die Berliner den Mann „hassen“, den sie kurz zuvor verehrten, und beschreibt ein Plakat, das in der Nacht an einer Ruine aufgehängt worden war und auf dem stand: „Das danken wir dem Führer!“ – ein Ausdruck, den Goebbels für einen ganz anderen Zweck geprägt hatte (ebd.: 18). Ein Chemiker kommentiert eine Rede von Goebbels, die dieser am Vorabend von Hitlers letztem Geburtstag hielt: „Es war das die tollste Rede, die er je gehalten hat. Sie hatte verzweifelte Ähnlichkeit mit dem Verhalten eines vor dem Tode stehenden Tuberkulosen, der glaubt, daß sich jetzt alles zum Besten wenden wird. Oder hat hier ein völlig Wahnsinniger gesprochen?“ (ebd.: 42).

Paula Nemeskei aus Nürnberg fragt sich angesichts der Kämpfe um die Stadt am 20. April: „muß denn unsere an sich schon so zerstörte Stadt noch ganz kaputt werden?“; und sie kommentiert: „Aber die Nationalsozialisten bleiben sich treu: alles noch opfern, weil sie selbst ihr Ende sehen“ (ebd.: 52). Marthel Kaiser aus dem westfälischen Neheim schreibt zu Hitlers Geburtstag: „Verfluchen werden viele, ja fast alle Deutschen heute diesen Tag, und wie waren wir einmal so froh darüber. Wie hat sich doch alles gewandelt. Was hat dieser eine Mann doch alles getan. Gutes, ja gewiß, es ist nicht zu leugnen, daß auch manches dabei war, was gut war und idealistisch, aber das Gute wiegt nicht all das Furchtbare auf, was schließlich und endlich über unser deutsches Volk gekommen ist, durch diesen Mann.“ Sie wirft Hitler vor, das Volk betrogen zu haben (ebd.: 83f.). Der Schriftsteller Erich Weinert bezeichnet aus dem Moskauer Exil die Deutschen als „zerrissenes, entwurzeltes, irregeführtes Volk“ (ebd.: 215). Goebbels um Hitlers Geburtstag gehaltene Reden, in denen er prophezeite, Deutschland werde nach dem Krieg „in wenigen Jahre aufblühen wie zuvor“, es werde zu „Ordnung, Frieden und Wohlstand“ kommen (ebd.: 10), sowie seine Loyalitätserklärung an Hitler im Namen der gesamten deutschen Nation (ebd.: 41) stoßen auf Unglauben (ebd.: 15, 369) und Hohn (ebd.: 16). Goebbels wird als „der oberste Märchenerzähler“ (ebd.: 47) und Hitler selbst entweder als wahnsinnig oder als „der leibhaftige Satan“ bezeichnet (ebd.: 240). Diese Beispiele illustrieren eine Dimension von Kempowskis beherrschendem Thema in „Abgesang ’45“: die Deutschen als Opfer. Und anders als bei der Frage, was die Deutschen über die Konzentrationslager wussten, bleibt diese Vorstellung weitgehend unangefochten.

Die Herausstellung des Opferstatus der Deutschen wird durch Kempowskis Methode befördert: Indem er Momentaufnahmen aus dem Leben der Menschen präsentiert, schneidet er deren Kontext der Vergangenheit und Zukunft ab. Dies zeigt sich zum Beispiel in dem Auszug von Hanns Lilje zum 20. April 1945. Lilje, der seit 1944 wegen Verbindungen zu Gruppen, die sich gegen die Gleichschaltung der Kirchen durch die Nazis wandten, inhaftiert war, erlebte den Einmarsch der Amerikaner in Nürnberg von seiner Zelle in einem Gestapo-Gefängnis aus. Er kommentiert die Befreiung der Stadt durch die Amerikaner mit: „Da ist die jäh aufsteigende Bitterkeit, daß es Fremde sein müssen, die uns das kostbare Gut der Freiheit wiedergeben, das die eigenen Volksgenossen uns geraubt haben“ (ebd.: 53). Was in dieser Momentaufnahme, die das Leitmotiv von den Deutschen als Opfer des Nationalsozialismus wiederholt, nicht berücksichtigt wird, ist Liljes Haltung gegenüber den Nazis vor dem Krieg. Im Jahr 1932 hatte er deren bevorstehende Machtübernahme begrüßt: „Es ist mit großer Bestimmtheit zu erwarten, daß der Nationalsozialismus noch im Laufe dieses Jahres, vermutlich schon im Frühjahr, in irgendeiner Form an der Regierung beteiligt wird. Die Frage, ob das wünschenswert ist, ist mit Ja zu beantworten.“ (Lilje 1932: 72) 

Es geht dabei auch nicht einfach darum, dass die von Kempowski präsentierte Momentaufnahme die frühere Unterstützung des Nationalsozialismus nicht berücksichtigt. Nach dem Krieg wurde Lilje Mitglied im Rat der Evangelischen Kirche in Deutschland (EKD), und er war Unterzeichner des „Stuttgarter Schuldbekenntnisses“ des EKD-Rates 1945, in dem die EKD eine gewisse Verantwortung für den Nationalsozialismus übernahm und Selbstkritik übte, diesem nicht mutiger entgegengetreten zu sein (vgl. Besier, Sauter 1985). Liljes Vor- und Nachkriegspositionen bilden somit eine Herausforderung für die Opferperspektive in dem von Kempowski gewählten Ausschnitt.

„Abgesang ’45“ stellt die Deutschen auch als Opfer der vorrückenden feindlichen Armeen dar, insbesondere der Russen, die plündern und deutsche Frauen vergewaltigen (Kempowski 2005a: 37, 115, 282, 364).\endnote{Kempowski selbst hatte solche Szenen bei Kriegsende in Rostock miterlebt (vgl. Hempel 2004: 61).} Deutsche sind Opfer der alliierten Luftangriffe, die die Städte in Schutt und Asche legen (ebd.: 43, 70, 128). In den letzten Kriegstagen wollen nur noch die wahnhaften Naziführer weiterkämpfen (ebd.: 10, 101, 129, 214), während die unteren Ränge nicht mehr an einen Sieg glauben (ebd.: 14, 16, 20) und Vernunft walten lassen, indem sie kapitulieren und die Städte übergeben wollen, um weiteres sinnloses Sterben zu verhindern (ebd.: 52, 69). Während überzeugte Nazis die Deutschen für ihre „Selbstbeherrschung“ und „Disziplin“ loben (ebd.: 76), berichten andere vom Absturz ins Chaos, betrunkenen Offizieren (ebd.: 74) und Soldaten, die „unter der Uniform einen Zivilanzug tragen“ (ebd.: 20).

Kempowskis Methode, kurze Auszüge zu verwenden, tendiert auch dazu, das darin beschriebene Leid zu fragmentieren und zu dekontextualisieren. Während Kempowski in mehreren früheren „Echolot“-Bänden aus Victor Klemperers Tagebüchern zitiert, wie dieser als Jude in Nazideutschland immer restriktiveren antisemitischen Gesetzen unterworfen war (z. B. Kempowski 2002: 568f.), zitiert er in „Abgesang ’45“ Tagebucheinträge von Klemperer über die Entbehrungen des Alltags gegen Ende des Krieges und über seine Erleichterung und Dankbarkeit, überlebt zu haben: „[…] all das Elend unserer Situation: die Enge, die Primitivität, der Schmutz, die Zerfetztheit der Kleidung, des Schuhwerks, der Mangel an allem und jedem (wie Schnürsenkel, Messer, Verbandsstoff, Desinfektionsmittel, Getränk ...).“ (Kempowski 2005a: 266)Dies liest sich wie viele andere Passagen in „Abgesang ’45“, die von deutschen Zivilisten stammen, und der Leser erfährt nichts über den Kontext, der Klemperer von seinen Leidensgenossen unterscheidet. Während Klemperers Tagebücher in ihrer Gesamtheit dokumentieren, wie er als Jude seine Stelle an der Dresdner Universität verlor, er mit seiner Frau ins Ghetto umziehen musste und er trotz seiner Angina schwere körperliche Arbeit bei eisiger Kälte verrichten musste (Klemperer 2001: 8, 21, 27, 39, 44), trägt der von Kempowski im „Abgesang ’45“ zitierte Auszug zu einem undifferenzierten Bild einer Gemeinschaft von Deutschen bei, die in den letzten Kriegstagen leiden. \endnote{Susanne Vees-Gulani (2003: 109) kritisiert Kempowskis Vorgehensweise in „Das Echolot: Fuga Furiosa“, dass die Collage „unausgewogen“ sei, da alle Quellen als gleichwertig gesetzt werden, aber nur eine jüdische Stimme unter ihnen sei, nämlich die von Victor Klemperer.}

Das Thema des deutschen Leidens war für Kempowski schon lange mit seinem Projekt der Sammlung deutscher Autobiographien verbunden, allerdings auf ambivalente Weise: In einer Tagebuchnotiz vom 12. April 1987 reflektiert er über die Wirkung des Archivs auf diejenigen, die ihr autobiographisches Material einreichen: „Der einzelne Einsender erfährt eine späte Genugtuung, er weiß, daß sein Leiden wenigstens statistisch zu Buche schlägt. Sein Name ist im Himmel geschrieben“; er weist aber auch darauf hin, dass ein Teil des Materials ein Element von Selbstmitleid enthalte (Kempowski 2005b: 104). Kempowski gibt einen weiteren Hinweis zu seiner Motivation für die Arbeit an „Das Echolot“, wenn er sich fragt, warum er so viel Energie in das Projekt stecke, und zu dem Schluss kommt, dass sein Einsatz dem Bedürfnis entspringt, einer ganzen Generation Gerechtigkeit zu verschaffen: „Ich habe den Eindruck, dass man der Generation, die in diese Zeit hineingeboren war, nicht gerecht geworden ist.“ (ebd.: 235)

Bisher betrachtete ich, was Kempowski als Sammler und Chorleiter in „Abgesang ’45“ aufgenommen hat und wie er die Texte um das Thema des deutschen Leidens herum auswählte und anordnete. Kempowski selbst äußert sich kaum zu den Methoden und Prinzipien, die seinem großen Projekt zugrunde liegen: Die Einleitung zum ersten Band von „Das Echolot“ trägt den Titel „Statt eines Vorworts“ und ist nur eine Seite lang (Kempowski 1993: 7). In „Culpa“, seinem Tagebuch, das zeitgleich mit „Abgesang ’45“ erschien, äußert er sich eher zu praktischen Aspekten der Einrichtung seines Archivs und der Arbeit an „Echolot“. Ausführlichere Reflexionen über die Motive, die dem Werk zugrunde liegen, finden sich kaum. Als er sich im November 1992 daran setzt, einige Notizen für ein „Vorwort“ zu den ersten vier Bänden zu schreiben, lautet sein Tagebucheintrag: „Ich machte heute ein paar Notizen für das ‚Echolot‘-Vorwort. Sehr schwierig, das liegt mir gar nicht. Hildegard [Kempowskis Frau] meint, ich spreche wie ein Bauer, wenn ich mich theoretisch über etwas verbreite. – Ja, das Theoretisieren liegt mir nicht.“ (Kempowski 2005b: 243)

In Ermangelung direkter Äußerungen Kempowskis lässt sich nur aus der Art und Weise, wie er die Texte, die in das Projekt eingegangen sind, ausgewählt und bearbeitet hat, auf seine Motive schließen. Aus diesem Grund wende ich mich im Folgenden Kempowski \textit{Biografienarchiv} zu, das die Originalversionen der Texte enthält, aus denen ein großer Teil von „Echolot“ aufgebaut ist.

\section{Originalmaterial und Authentizitätsfrage}

\noindent Geht man im Entstehungsprozess noch einen Schritt weiter zurück und betrachtet das ursprüngliche Material, das in das Archiv gelangte, sowie die Begleitdokumente, so zeigt sich, dass einige der Kempowski zugesandten und in seinem Projekt verwendeten Berichte bereits zuvor vom jeweiligen Verfasser oder seiner Familie bearbeitet worden waren. Besonders deutlich wird dies in der Originaldokumentation zum Bericht des Obersteuermanns Fritz Bösel über eine Flottille kleiner Boote, die im nordpolnischen Marinehafen Hela eintraf, um deutsche Soldaten nach Westen zu transportieren (Kempowski 2005a: 340ff.). Vor und nach diesem Bericht schildern andere Soldaten, wie sie sich unter Artilleriebeschuss in die Boote drängten und aufs offene Meer flüchteten, wo ihnen mitgeteilt wurde, dass der Krieg zu Ende sei. Kempowski präsentiert die Berichte in diesem Teil seines Buches als Fluchtgeschichten in letzter Minute.

Das Originalmaterial, das die Familie Bösel 1991 Kempowski für dessen Archiv übergab, besteht hauptsächlich aus einem längeren undatierten Typoskript mit dem Titel „Papas Odyssee im 2. Weltkrieg“,\endnote{Der „Papa“ des Titels ist der Vater von Fritz Bösel (Akademie der Künste, Berlin, Kempowski-Biografienarchiv, Nr. 2114/1 – im Folgenden werden die Materialien aus Kempowskis  Biografienarchiv zitiert mit: AdK, Kempowski-Biografien).} ein Titel, der bereits auf eine Verarbeitung des Erlebten zu einer Heldenerzählung hinweist. In einem Brief vom 27. Februar 1994 erklärt Fritz Bösel, dass er und sein Bruder seit zwölf Jahren an der „Familien-Chronik“ arbeiten. Wir haben es hier mit einem Beispiel für eine Familie zu tun, die ihre eigene Geschichte produziert – die in Kempowskis Archiv dann als „Kriegs- und Familienchronik des Vaters und der Söhne Bösel“ bezeichnet wird. In einem auf Weihnachten 2002 datierten Begleitschreiben ermutigt Fritz Bösel Kempowski gegenüber seinem Verleger zu dem „Echolot“-Projekt: „Ich würde aber die Hoffnung nicht aufgeben, da inzwischen die Leiden der Deutschen in diesem schrecklichen Krieg sogar von Günter Grass entdeckt worden sind\endnote{Der Verweis bezieht sich auf „Im Krebsgang“, Günter Grass’ Novelle von 2002 über die Versenkung der „Wilhelm Gustloff“ durch ein sowjetisches U-Boot, bei der rund 9.000 Menschen ums Leben kamen.} und wünsche Ihnen daher viel Glück.“ Bösels Hinweis auf das deutsche Leiden macht die Familien­erinnerung umso brauchbarer.

Ein zweites Beispiel zeigt, wie Kempowski Passagen eines Originaltextes auslässt, um ihn in seine Agenda einzupassen. In dem Abschnitt über den 8. und 9. Mai 1945 in „Abgesang ’45“ zitiert Kempowski die Beschreibung einer Gruppe ausgemergelter ehemaliger Häftlinge des KZ Mauthausen, die in Linz am Straßenrand kochen. Der Autor des Berichts ist Günter Cords, der im Register des Bandes als Mitglied eines SA-Musikzuges aufgeführt ist. In der in „Abgesang ’45“ veröffentlichten Version erzählt Cords, wie die Häftlinge seinen Konvoi kontrollierten und zwei Kisten vom Wagen holten, obwohl die Fahrer angaben, im Auftrag der US-amerikanischen Armee unterwegs zu sein. Als Cords weiterfahren will, hockt sich ein polnischer Häftling auf seinen Wagen. Cords kommentiert: „Sein Kopf, der auf einem dünnen Hals saß, erinnerte eher an einen Totenkopf oder eine Fratze als an einen lebenden Menschen. Unter der Sträflingskleidung zeichneten sich die dürren Knochen ab […]. Es kostete mich Überwindung, ihn anzuschauen.“ Der polnische Gefangene erzählt Cords, dass die Amerikaner an diesem Morgen Tausende von Ausländern befreit haben, die in Mauthausen inhaftiert waren, und er berichtet von den „unglaublichen Zuständen“ dort (Kempowski 2005a: 400f.).

In Cords archiviertem Originalbericht findet sich ein 188-seitiges Typoskript mit dem Titel „Abenteuer eines SA-Musikschülers. Mit längerem Nachspiel“.\endnote{AdK, Kempowski-Biografien 2351.} Auch hier deutet der Titel darauf hin, dass der Autor seinen Bericht bereits zu einer Erzählung geformt hat. Dies wird durch die Tatsache unterstrichen, dass es sich bei dem Typoskript um ein zusammenhängendes, undatiertes Dokument handelt und nicht um eine Sammlung von Briefen und Notizen, die zur Zeit der von Cords beschriebenen Ereignisse entstanden sind. In einem Begleitschreiben vom 18. November 1988 an Kempowski gibt Cords einige biographische Informationen zu sich: Er wurde 1928 geboren, besuchte mit vierzehn Jahren eine SA-Musikschule und war vom 17. bis zum 20. Lebensjahr in Kriegsgefangenschaft.

Ein Vergleich der Originalfassung (S. 77 des Typoskripts) mit dem von Kempowski zitierten Auszug zeigt, dass in Kempowskis Version Sätze weggelassen wurden. Im Original wird erzählt, wie Cords über die Kontrolle durch die KZ-Häftlinge empört war und meinte, er solle sie lieber Banditen nennen. Und im Unterschied zu Kempowskis Version fährt Cords in seinem Originaltyposkript nach dem Kommentar, dass es ihm schwerfiel, den polnischen Mann anzusehen, fort: „Da sahen wir Deutschen doch anders aus“ (S. 77).

In seinen editorischen Anmerkungen zu „Abgesang ’45“ schreibt Kempowski, dass die von ihm aufgenommenen Texte „in den meisten Fällen nicht gekürzt“ wurden. „Auslassungen am Anfang oder Ende eines in sich geschlossenen Textes habe ich in der Regel nicht angezeigt. Hingegen habe ich Streichungen innerhalb eines Textes durch [...] kenntlich gemacht“ (Kempowski 2005a: 455). In diesem speziellen Fall wich Kempowski jedoch von dieser Praxis ab und verzichtete auf die Kennzeichnung der Auslassung der Passage mit Cords’ Empörung und dessen Ansicht, dass die Häftlinge eher wie Banditen aussahen, und er strich Cords’ Bemerkung, dass die Deutschen anders aussahen als der junge Pole.\endnote{In einer Tagebuchnotiz vom September 1992 vermerkt Kempowski eine Rezension zu einer neuen Ausgabe der Goebbels-Tagebücher, in der kritisiert wird, dass die Ausgabe zu viel weggelassen habe. Kempowski ist besorgt, dass er in seinen eigenen Texten „Streichungen nicht kenntlich gemacht habe“ (Kempowski 2005b: 229f.).}

Kempowski übernimmt auch nicht die Schlussfolgerung von Cords, dass der Mann etwas falsch gemacht haben muss, sonst wäre er nicht eingesperrt worden (S. 78 des Typoskripts). Kempowskis veröffentlichter Auszug endet dort, wo der Pole von den schrecklichen Zuständen in Mauthausen berichtet. Im Original-Typoskript fährt der Mann fort: „Nix Essen, bloss Wassa und zwei Kartofel Pell für einen Tag. SS nix gutt.“ (S. 78). Cords reagiert böse auf diese Andeutung, dass die SS den Mann misshandelt habe: „Mir war der Kerl unsympathisch.“ Cords beschließt seine Sicht auf die Konzentrationslager, indem er sagt, er habe immer geglaubt, dass sie für Volksfeinde, Kommunisten, Zeugen Jehovas und diejenigen, die am „Verrat des 20. Juli“ beteiligt waren, gewesen seien (S. 78).

Die unmarkierten Streichungen Kempowskis und seine Auswahl aus dem Originaltext führen insgesamt zu einer Abschwächung des Eindrucks, den der SA-Musiker macht: Cords erscheint als junger Beobachter, der sich über das, was er sieht, aufregt, und nicht als jemand, der sich über das Verhalten des Häftlings empört, der eine Abneigung gegen diesen empfindet, weil er die SS kritisiert, und der das Gefühl hat, dass er etwas falsch gemacht haben muss, um überhaupt in Mauthausen gelandet zu sein. Die Streichungen und Auslassungen stellen die Authentizität der in „Abgesang ’45“ veröffentlichten Texte in Frage, die in der Rezeption des „Echolot“-Projekts oft gelobt wird.\endnote{So bezeichnet etwa Drews „Echolot“ als „ein Lesebuch für die Nation, ohne einen Erzähler, nur mit einem Arrangeur von nicht-künstlerischen, von authentischen Texten“ (Drews 2006: 48).}

Der Eindruck, den Cords in der in „Abgesang ’45“ veröffentlichten Passage hinterlässt, ist weit entfernt vom generellen Ton seiner übrigen Erinnerungen. Er erzählt von Herrn Bullerbach, einem Propagandaoffizier an seiner Schule, der „Rassenschande“ betrieb. Cords beschreibt dies als schändliches Verhalten, da Bullerbach seinen Schülern immer gesagt hatte, dass die germanische Rasse die edelste und beste sei (S. 2, datiert von September 1944). Cords notiert später, dass nach dem „Verrat vom 20. Juli“ mehr als dreihundert Volksfeinde hingerichtet und einige Tausend in Konzentrationslager geschickt wurden. Cords ist voller Bewunderung für die Leistungen der Gestapo, aber entsetzt über das Ausmaß des Verrats im ganzen Land (S. 30).

Das nächste Beispiel veranschaulicht, wie Kempowski Texte auswählte, die sein eigenes Grundmuster der Kriegserfahrung tendenziell unterstützen: die Trennung zwischen der breiten Masse der deutschen Bevölkerung auf der einen Seite und der kleinen Gruppe der Nazi-Elite und Nazi-Fanatiker auf der anderen Seite. Im auf den 8./9. Mai 1945 datierten Abschnitt präsentiert Kempowski u. a. einen Auszug aus den Erinnerungen von Dr. Kuno Gerner über seine Erfahrungen in vierzehn Monaten Kriegsgefangenschaft im Camp Howze in den USA. Gerner schildert in leicht feindseligem Ton seine Behandlung durch die Lagerwachen und erzählt abschließend, wie den Gefangenen mitgeteilt wurde, dass Deutschland vor den Westmächten kapituliert habe, nicht aber vor Russland. Gerner berichtet, dass die „Naziköpfe“ unter den Gefangenen sich darüber freuten, da sie glaubten, dass sie nun von den Amerikanern im Kampf gegen die Russen eingesetzt werden (Kempowski 2005a: 370f.).

Gerners ursprüngliche Memoiren bestehen aus einem 117-seitigen Typoskript, das nach Angaben des Autors im Jahr 1946 verfasst wurde.\endnote{AdK, Kempowski-Biografien 1376} Es ist ein weitgehend sachlicher Bericht über das, was Gerner in den USA sah, mit seitenlangen Details über die amerikanische Landwirtschaft, Gebäude und das Kriegsgefangenenlager selbst. Abgesehen von diesen Details stellt Gerner jedoch fest, dass die Kriegsgefangenen immer noch überwiegend für Hitler sind. Gerner schickte Kempowski auch einen Artikel, den er 1946 für die Lagerzeitung „Die neue Saat“ geschrieben hatte. Der Tenor des Artikels ist, dass es Hitlers Krieg gewesen sei, und Gerner erklärt: „Wir haben es erlebt, wir sind Zeugen, wie ein Krieg aus den dämonisch-bösen Kräften eines Menschen, einer Clique sich zusammenbraute, und wie er dann vom Zaun gebrochen wurde, in frevelndem Wahnsinn. Wenn die erbärmlichen Gestalten in Nürnberg heute beteuern, alles getan zu haben, was in ihren Kräften stand, um diesen Krieg zu verhindern, so scheinen sie nur uns und die Welt für dumm zu halten. Das war Hitlers Krieg, und dieser Krieg war sein Programm! Dieser Krieg ist dem letzten anständigen Menschen abscheulich und verhaßt geworden.“\endnote{Ebd., Die neue Saat 1946, S. 6.}

Diese Interpretation des Krieges deckt sich mit Kempowskis eigener Sicht in „Abgesang ’45“, sie hilft zu erklären, warum Kempowski Gerners Bericht auswählte. Gerners Hinweis darauf, dass die Mehrheit der Kriegsgefangenen für Hitler war, passt jedoch nicht zu der einfachen Unterscheidung zwischen Deutschen und Nazis und taucht entsprechend auch in Kempowskis Auswahl nicht auf.

Wenn Kempowski pro-nazistische Memoiren erhält, kommen sie manchmal über Nachkommen, die sich von den Eltern distanzieren. Dies ist der Fall bei dem Auszug in „Abgesang ’45“, in dem Otto Faust, ein Kriegsgefangener im Lager Szallias-Pilz bei Riga, erzählt, wie er die Nachricht von der bedingungslosen Kapitulation Deutschlands erfuhr: „Es war für mich sehr schwer. Als Idealist zog ich in diesen Kampf, ob als Soldat oder Nationalsozialist, und ein so gewaltiges Ringen nahm so einen bitteren Abschluß.“ Die Antifaschisten, die Faust auch als „Vagabunden“ bezeichnet, dagegen freuten sich über die Kapitulation Deutschlands. Faust beschreibt weiter, wie einige Kriegsgefangene im Lager an Unterernährung starben (Kempowski 2005a: 346).

Das Originaldokument wurde Kempowski 1985 von Armin Peter Faust, dem Sohn von Otto Faust, zugesandt.\endnote{AdK, Kempowski-Biografien 1153.} Der verwendete Auszug entspricht dem Original (S. 19), und Armin Peter Faust erklärt in einem Begleitschreiben, sein Vater habe den Text 1953 auf der Grundlage von Notizen und Briefen verfasst. Aber der Sohn fügt hinzu: „Es ist der subjektive Bericht eines Menschen, der einem System diente, das ich in meinem Text – 30 Jahre später – als ‚verbrecherisch‘ bezeichnet habe.“\endnote{Ebd., Schreiben vom 18.9.1985. Der Text, auf den sich der Sohn bezieht, befindet sich nicht in den Archivunterlagen.} Während Kempowski berechtigt war, den Text des Vaters so zu verwenden, wie er es wünschte, ist es klar, dass die widersprüchlichen Haltungen zwischen den Generationen, die für den Sohn wichtig sind, im „Echolot“-Projekt verloren gehen.

Als eine Variante des Falls von zwei Generationen einer Familie, die miteinander im Streit liegen, kann auch ein Individuum gesehen werden, das über einen längeren Zeitraum in einem kritischen Dialog mit sich selbst steht. Das zeigt sich oft nicht in den Auszügen, die Kempowski in „Abgesang ’45“ aufnimmt, sondern erst in der Gesamtheit des Materials, das Kempowski von dieser Person zugesandt wurde.

Dies wird u. a. bei Verwendung der Memoiren deutlich, die Harro Ketels an Kempowski schickte. Sie bestehen zum einen aus einem undatierten Dokument mit dem Titel „Im Kessel von Demjansk. Auszug aus meinem Kriegstagebuch 13. Jan. – 25. April 1942“ und einem Vorwort des Autors.\endnote{AdK, Kempowski-Biografien 6763/1-2 (Mappe 1)} Das Dokument hat eine komplexe Struktur, da es eine Mischung aus Tagebuchauszügen aus dem Jahr 1942 und zu einem späteren Zeitpunkt in eckigen Klammern hinzugefügten Kommentaren ist. Ein solcher Kommentar lautet zum Beispiel: „Heute sehe ich längst alles von einem anderen Standpunkt aus als damals. Ich war 27 Jahre alt und machte mir 1942 noch kaum kritische Gedanken über Sinn und Berechtigung eines solchen Krieges!“ Weiterhin schickte Ketels Kempowski einen weiteren Text mit dem Titel „Meine Einstellung zum Nationalsozialismus, zu Hitler und dem Krieg – Antwort auf kritische Fragen meiner Kinder von 1980“.\endnote{AdK, Kempowski-Biografien 6763/1-2 (Mappe 2).}

In „Abgesang ’45“ zitiert Kempowski aus dem Kriegstagebuch Ketels einen Bericht vom April 1945 über einen Vortrag zum Thema „Die wahre Zukunft Europas“. Ketels kommentiert, dass der Vortrag ihn „zwang“, seine Ansichten zu überdenken, da er nun sah, dass Nationalsozialismus und Bolschewismus viele Gemeinsamkeiten hätten (Kempowski 2005a: 145). Der 1915 geborene Ketels war Oberleutnant in Kurland, und der Vortrag, der als „Sonderdruck für Kommandeure“ gedruckt worden war, war von Professor Baumgarten, der ihn ursprünglich in Königsberg gehalten hatte, „als die Russen schon vor der Stadt standen“. Kempowskis Kontext in „Abgesang ’45“ ist der Vormarsch der sowjetischen Truppen und der Rückzug der Deutschen.

In anderen, von Kempowski nicht verwendeten Teilen seines Kriegstagebuchs berichtet Ketels, dass der „Führer“ jeden Rückzug verboten habe, die Truppen sich aber dennoch zurückziehen. Soldaten müssten mit Waffengewalt bedroht werden, damit sie auf ihren Posten bleiben (S. 2f.). Ketels selbst ist kein widerwilliger Soldat, er trotzt den Aufforderungen der Russen zur Kapitulation (S. 6) und ist erfreut über die hohen täglichen Verluste des Feindes (S. 9). Obwohl er später kommentiert, er habe kaum über den Sinn und die Rechtfertigung des Krieges nachgedacht, hat er damals Zweifel an dem Befehl, russische Kommissare zu töten, da er befürchtet, dass sich dies auf die Behandlung anderer Russen in Situationen auswirken könnte, in denen es einfacher wäre, sie zu erschießen als sie gefangen zu nehmen (S. 9).

Anders als der in „Abgesang ’45“ zitierte Auszug vermittelt das Kriegstagebuch in seiner Gesamtheit den Eindruck eines Autors im kritischen Dialog mit sich selbst. Dieser Eindruck wird noch verstärkt, wenn man die 1980 verfasste Reflexion Ketels’ über seine Einstellung zum Nationalsozialismus, zu Hitler und zum Krieg betrachtet. Ketels schrieb den Text als Antwort auf kritische Fragen seines Sohnes, u. a.: „Unter welchen Voraussetzungen, mit welcher Geistes- und Gefühlshaltung, unter welchen Wertvorstellungen oder Lebenszielen sind Menschen fähig oder fühlen sich bemüßigt, Krieg zu führen und zu zerstören? Konkret: Mit welchen Vorstellungen, Gedanken, Gefühlen und Ängsten zogst Du in den Krieg?“ (S. 3).

Der Vater versucht, die Fragen anhand von Auszügen aus seinem Tagebuch zu beantworten, das die Jahre 1934 bis 1948 umfasst, also einen weitaus größeren Zeitraum als der Tagebuchauszug, den Ketels an Kempowski geschickt hat. Der erste Tagebucheintrag, den Ketels zitiert, datiert vom 22. Mai 1935 und berichtet von einer Rede Hitlers, die Ketels besonders anregend fand: „Hitlers Rede vom 21. Mai wird in die Geschichte eingehen. Sie gehört zu haben, muß jeden Deutschen mit Stolz erfüllen. Und doch höre ich Leute sagen, das war ja wieder einmal dasselbe. Ihnen ist nicht zu helfen. Ist es doch gerade das Große, daß wir heute nur eine einzige Forderung an die Welt haben: Gleichberechtigung und Frieden. Von unserer unerschütterlichen Absicht, das erste zu erringen, und von dem fast übermenschlichen Bestreben, das zweite zu sichern, legt diese Rede ein großes Zeugnis ab. Hinter ihr steht nicht nur die ehrliche Gesinnung, sondern die realisierbare Tat. Die Welt muß sich scheiden an dieser Rede.“ (S. 1)

Im Jahr 1980 kommentiert Ketels dieses damalige jüngere Ich: „Als ich dies schrieb, war ich gerade zwanzig Jahre alt und studierte Geschichte, Deutsch und Religion im 4. Semester. Dreizehn Jahre später – nach Rückkehr aus russischer Gefangenschaft –, als der ganze Spuk des sogenannten Dritten Reiches vorüber war, begann ich alles zu lesen, was an Quellen, Berichten und Memoiren über jene Zeit zu finden war. Nun erfuhr ich überhaupt erst von den ungeheuren Verbrechen Hitlers. Daß es einen Kreis des Widerstandes gab, hörte ich zum erstenmal nach dem mißglückten Attentat am 20. Juli 1944 – natürlich aus der Sicht der nationalsozialistischen Propaganda.“ (S. 3)

Ketels fährt fort, dass manche Menschen im Nachhinein behaupten, nichts von den Gräueltaten des NS-Systems gewusst zu haben, und er sagt, dass dies auf die meisten Menschen nicht zutrifft. Er selbst zum Beispiel habe im November 1938 gewusst, dass die jüdische Synagoge in Bielefeld in Brand gesteckt worden war. Im Sommer 1940 habe er gehört, dass einige Geisteskranke weggebracht und wahrscheinlich umgebracht worden waren, und er habe Juden in Litauen gesehen, die an den Bahnlinien arbeiteten, darunter Frauen in Pelzmänteln. Im August 1943 hatte er das Geständnis eines hochdekorierten Infanterieoffiziers gehört, der 1939 in Polen als Dolmetscher für die SS gearbeitet hatte (S. 1f.). Der ältere Ketels beschreibt diese Informationen als Mosaiksteine, die sich in seinem Kopf nie zu einem Gesamtbild zusammengefügt haben, und er ist schockiert, dass er sich damals nicht mit den Ungerechtigkeiten, von denen er gehört hatte, auseinandergesetzt hat. Ungläubig stützt er den Kopf in die Hände, als er das Manuskript eines Vortrags liest, den er 1945 hielt, als die durch Hitlers Politik verursachte Katastrophe für jedermann sichtbar war. In diesem Vortrag hatte er gesagt, wenn Deutschland den Krieg verliere, sei es das Beste für jeden einzelnen Mann, bis zum Tod zu kämpfen wie die Nibelungen in Etzels Halle (S. 4). Rückblickend beschreibt Ketels die Jahre 1933 bis 1935 als eine heroische Zeit, in der der jugendliche Idealismus von Hitler und der Parteiführung missbraucht wurde. Auch die intellektuelle Elite und die hohen Generäle seien größtenteils getäuscht worden oder waren verängstigt (S. 2).

Ketels erklärt, wie er um 1935 zum Theologiestudium wechselte. Die Behandlung der Kirche und des Christentums durch die Nazis beunruhigte ihn: Im Herbst 1937 waren 200 Pfarrer inhaftiert. Ketels schloss sich einer kirchlichen Studentengruppe in Halle an, in Opposition gegen die von den Nazis eingesetzte Kirchenleitung, und er protestierte 1939 in einem Brief an das für die Kirchen zuständige Ministerium, mit Kopie an Göring, als die Kirchliche Hochschule Bethel von der Gestapo geschlossen wurde (S. 7). Gegen Ende des Krieges wurde er verdächtigt, illegale Gruppen zu organisieren, und verhört (S. 15).

In seinen Betrachtungen über den Nationalsozialismus, Hitler und den Krieg von 1980 weicht Ketels von seinem Schema, aus seinem Tagebuch zu zitieren und dann zu kommentieren, ab, wenn die Tagebucheinträge zu grausam sind. Er schreibt, dass sein Tagebucheintrag für den 3. Juli 1941 Details von zwei Augenzeugenberichten über die brutale und grausame Behandlung von Juden in Kaunas durch Litauer im Monat zuvor enthält. Ketels schreibt über seinen Tagebucheintrag: „Es folgen fürchterliche Einzelheiten, die ich damals aufschrieb und inzwischen längst vergessen hatte.“ (S. 14).\endnote{Für einen Augenzeugenbericht über die Ermordung der Juden in Kaunas vgl. Ullrich 2000.} Im Rückblick ist Ketels besonders entsetzt darüber, dass diese Ereignisse am 26. Juni 1941 mit einem deutschen Vormarsch zusammenfielen, was ihn zu der Annahme veranlasst, dass die Führung des Zweiten Armeekorps von den Geschehnissen gewusst haben musste und an ihnen beteiligt war (S. 14f.). Ketels kommt zu dem Schluss, dass er erst nach dem Krieg begriffen hat, dass Hitler die deutschen Soldaten in seinem „ideologischen Vernichtungskrieg“ missbraucht habe (S. 15).

Insgesamt sind Ketels’ Überlieferungen an das Kempowski-Archiv durch Komplexität, Widersprüchlichkeit, Selbstanklage und Selbstentlastung gekennzeichnet. Die Verflechtung von Tagebucheinträgen aus den Kriegsjahren mit kritischen Einschätzungen seines älteren Ichs aus der Nachkriegszeit deutet darauf hin, dass Ketels eher das Endprodukt eines Langzeitprojekts als eine Sammlung aus seinen damaligen Tagebüchern übergab. Die bohrenden Fragen seines Sohns veranlassten Ketels, seine Kriegstagebücher erneut zu lesen und über sein eigenes Versagen und das von Schlüsselgruppen in den Jahren der Naziherrschaft nachzudenken. Ketels schwankt zwischen dem Eingeständnis, dass er von den Gräueltaten der Nazis schon sehr kurz nach deren Geschehen wusste, sowie der Zurückweisung der Behauptung anderer, dass dies nicht der Fall war, und der Behauptung, dass er selbst erst nach dem Krieg davon wusste. Sein Schock darüber, dass er damals nichts unternommen hat, zeugt von einem Gefühl der persönlichen Verantwortung, aber er kommt auch immer wieder auf den Gedanken zurück, dass sein Idealismus und der anderer deutscher Soldaten von Hitler missbraucht wurde. Ketels protestierte gegen die Maßnahmen der Gestapo gegen die evangelische Kirche und geriet später unter Verdacht, doch er war ein funktionierender Soldat, der seine Kameraden aufforderte, notfalls bis zum Tod zu kämpfen. Betrachtet man seine nuancierte Schilderung als Ganzes, so erhält der Leser einen Einblick in eine konflikthafte Mentalität.

Auch wenn eine Fragmentierung der ursprünglichen Erzählungen in einer Sammlung von Auszügen wie „Abgesang ’45“ unvermeidlich ist, durchkreuzt Kempowskis Präferenz für Momentaufnahmen und faktische Schilderungen von Ereignissen anstelle einer reflektierten Bewertung der individuellen Verantwortung die dem Projekt zugrunde liegende Motivation, wie sie vom Verfasser über einen längeren Zeitraum hinweg konzipiert wurde, und setzt sie außer Kraft.

\section{Zwischen Kollektiv und Tagebuch}

\noindent „Echolot“ als Ganzes wurde als Kempowskis später und finaler Triumph betrachtet, nachdem er aufgrund eines Großteils seines Frühwerks als „schreibender Schulmeister“ abgetan worden war (Wieser 1990). Die Kritiker von „Echolot“ waren in der Minderheit, und höchstens spekulativ wiesen sie auf die Möglichkeit einer politischen Voreingenommenheit hin, die Kempowskis Auswahl von Texten aus der Masse des ihm zur Verfügung stehenden Materials beeinflusste (vgl. z. B. Rehfeldt 2010: 375).

Die Analyse zeigt jedoch nicht nur, dass es sich bei dem an Kempowski gesandten Material oft nicht um eine Sammlung von Originalbriefen, Tagebüchern und Notizen aus den letzten Kriegsjahren handelt, wie es der Untertitel des „Echolot“-Projekts – „ein kollektives Tagebuch“ – und die Rezensenten suggerieren,\endnote{Iris Radisch (2005) schreibt in ihrer Rezension zu „Abgesang ’45“, dass Kempowski die Wirklichkeit in ihrem „Rohzustand“ darstellen wollte.} sondern vielmehr um Berichte, die von den Autoren und anderen Familienmitgliedern nach den Ereignissen, die sie beschreiben, und bevor sie zu Kempowski gelangten, zu Geschichten verarbeitet wurden. Die Analyse zeigt auch, dass „Abgesang ’45“ zwar einen inneren Dialog enthält, der etwas von der Heterogenität einfängt, die Assmann mit individuellen Erinnerungen verbindet, dass dieser Dialog aber letztlich zugunsten einer kollektiven Darstellung des deutschen Leidens durch einen systematischen Prozess der Auswahl, Musterung, Anpassung und Rekontextualisierung des Materials verloren geht. Dieser Prozess steht im Widerspruch zu dem Bestreben, die subjektive Erfahrung von Geschichte zu vermitteln. Kempowskis Methode richtet den Fokus nicht auf die einzigartige individuelle Erfahrung, sondern bearbeitet diese Erfahrung, um sie zu rehomogenisieren. Mit Assmann (2006: 157) gesprochen, gewinnen „kollektiv homogenisierende Impulse“ gegenüber „heterogenen individuellen Erinnerungen“ die Oberhand. Im Sinne von Kempowskis Untertitel für das „Echolot“-Projekt siegt das Kollektiv über das Tagebuch.

\textit{Aus dem Englischen von Thomas Möbius}

\printendnotes[custom]

\section{Literatur}
    \begin{bibdescription}
        \item Assmann, Aleida (2006): Der lange Schatten der Vergangenheit. Erinnerungskultur und Geschichtspolitik. München: C. H. Beck.
        \item Besier, Gerhard; Sauter, Gerhard (1985) Wie Christen ihre Schuld bekennen. Die Stuttgarter Erklärung 1945. Göttingen: Vandenhoeck \& Ruprecht.
        \item Damiano, Carla (2005): Walter Kempowski’s „Das Echolot“. Sifting und Exposing the Evidence via Montage. Heidelberg: Universitätsverlag Winter.
        \item Drews, Jörg (2006): „Die Dämonen reizen – und sich dann blitzschnell umdrehen, als sei nichts“. Über Walter Kempowski. In: Text+Kritik 169: Walter Kempowski, S. 44-52.
        \item Friedländer, Saul (1997): Nazi Germany and the Jews, Bd. 1: The years of persecution, 1933–1939. New York: HarperCollins.
        \item Friedländer, Saul (2007); Nazi Germany and the Jews, Bd. 2: The years of extermination, 1939–1945. New York: HarperCollins. 
        \item Fritzsche, Peter (2002): Volkstümliche Erinnerung und deutsche Identität nach dem Zweiten Weltkrieg. In: Jarausch, Konrad; Sabrow, Martin (Hrsg.): Verletztes Gedächtnis, Erinnerungskultur und Zeitgeschichte im Konflikt. Frankfurt a. M., New York: Campus, S. 75-97.
        \item Fulbrook, Mary; Rublack, Ulinka (2010): In Relation. The ‚Social Self’ and Ego-Documents. In: German History 28, S. 263-272.
        \item Hage, Volker (2000): „Das hatte biblische Ausmaße“. Interview mit Walter Kempowski. In: Der Spiegel Nr. 13, vom 27.03.2000, S. 264-268; URL: https://www.spiegel.de/kultur/das-hatte-biblische-ausmasse-a-58ab720d-0002-0001-0000-000016044569.
        \item Hage, Volker (2005): Der Chor der Stummen. In: Der Spiegel Nr. 7, 14.02.2005, S. 166-168.; URL: https://www.spiegel.de/kultur/der-chor-der-stummen-a-85f49a14-0002-0001-0000-000039367972. 
        \item Hempel, Dirk (2004): Walter Kempowski: Eine bürgerliche Biographie. München: btb.
        \item Hempel, Dirk (2006): Das Archiv der unpublizierten Autobiographien. In: Walter Kempowskis Archive. Hrsg. von der Kulturstiftung der Länder in Verbindung mit der Akademie der Künste. Berlin: Kulturstiftung der Länder, S. 43-60.
        \item Hempel, Dirk (2007): Kempowskis Lebensläufe. Akademie der Künste: Berlin.
        \item Jancke, Gabriele (2002): Autobiographie als soziale Praxis. Beziehungskonzepte in Selbstzeugnissen des 15. und 16. Jahrhunderts im deutschsprachigen Raum. Köln u. a.: Böhlau.
        \item Kempowski, Walter (1979): Haben Sie davon gewußt? Deutsche Antworten. Hamburg: Knaus.
        \item Kempowski, Walter (1993): Das Echolot. Ein kollektives Tagebuch, Januar und Februar 1943, Bd. I. München: Knaus.
        \item Kempowski, Walter (1999): Das Echolot. Fuga foriosa. Ein kollektives Tagebuch, Winter 1945; Bd. IV. München: Knaus.
        \item Kempowski, Walter (2002): Das Echolot. Barbarossa ’45. München: Knaus.
        \item Kempowski, Walter (2005a): Das Echolot. Abgesang ’45. Ein kollektives Tagebuch. München: Knaus.
        \item Kempowski, Walter (2005b): Culpa: Notizen zum „Echolot“. München: Knaus.
        \item Klemperer, Victor (2001): Ich will Zeugnis ablegen bis zum letzten: Das Jahr 1942. Hrsg. von Walter Nowojski. Berlin: Aufbau 2001.
        \item Lilje, Hanns (1932): Das politische Gesicht der Zeit. In: Evangelische Wahrheit 23, H. 4, S. 70-72.
        \item Niven, Bill (Hg.) (2006): Germans as Victims. Remembering the Past in Contemporary Germany. Basingstoke, New York: Palgrave Macmillan.
        \item Plöschberger, Doris (2006): Der dritte Turm: Die Tagebücher Walter Kempowskis. In: Text+Kritik 169: Walter Kempowski, S. 32-43.
        \item Radisch, Iris (2005): Phrasen, die keiner mehr kennt. In: Die Zeit Nr. 9, 24.02.2005.
        \item Rehfeldt, Martin (2010): Archiv und Inszenierung. Zur Bedeutung der Autoreninszenierung für Walter Kempowskis Das Echolot und Benjamin von Stuckrad-Barres Soloalbum. In: Hagestedt, Lutz (Hrsg.): Walter Kempowski. Bürgerliche Repräsentanz, Erinnerungskultur, Gegenwartsbewältigung. Berlin: de Gruyter, S. 369-390.
        \item Reich-Ranicki, Marcel (1994): Das Literarische Quartett. ZDF, 24.02.1994.
        \item Ritte, Jürgen (2009): Endspiele. Geschichte und Erinnerung bei Walter Kempowski, Dieter Forte und W. G. Sebald. Berlin: Matthes \& Seitz.
        \item Schirrmacher, Frank (1993): In der Nacht des Jahrhunderts: Walter Kempowskis „Echolot“. In: Frankfurter Allgemeine Zeitung, 13.11.1993; URL: https://www.faz.net/aktuell/feuilleton/walter-kempowskis-echolot-in-der-nacht-des-jahrhunderts-1489392.html.
        \item Schmitz, Helmut (2007a): Introduction: The Return of Wartime Suffering in Contemporary German Memory Culture, Literatur and Film. In: ders. (Hg.): A Nation of Victims: Representations of German Wartime Suffering from 1945 to the Present. Amsterdam, New York: Rodopi, S. 1-29.
        \item Schmitz, Helmut (Hg.) (2007b): A Nation of Victims: Representations of German Wartime Suffering from 1945 to the Present. Amsterdam, New York: Rodopi.
        \item Seibt, Gustav (2005): Nun schweigen an allen Fronten die Waffen. In: Süddeutsche Zeitung, 25.02.2005; URL: https://www.sueddeutsche.de/kultur/literatur-nun-schweigen-an-allen-fronten-die-waffen-1.880157-0\#seite-2.
        \item Ullrich, Volker (2000): So teuflisch, so unbeschreiblich entsetzlich. Ein einzigartiges Dokument: Die Aufzeichnungen der Helene Holzman über die Vernichtung der litauischen Juden. In: Die Zeit Nr. 47, 16.11.2000.
        \item Vees-Gulani, Susanne (2003): Trauma and Guilt. Literature of Wartime Bombing in Germany. Berlin, New York: de Gruyter.
        \item Werner, Hendrik (2005): Palast und Ballast der Erinnerung. In: Die Welt, 12.02.2005; URL: https://www.welt.de/print-welt/article424373/Palast-und-Ballast-der-Erinnerung.html.
        \item Wieser, Harald (1990): Der Abschreiber. In: Stern Nr. 3, 11.01.1990, S. 29-34.
\end{bibdescription}
\end{multicols*}
